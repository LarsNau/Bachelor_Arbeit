% ----------------------------------------------------------------------
%
%        Vorlage für Abschlussarbeiten am Lehrstuhl Informatik VII
%
%                   http://ls7-www.cs.uni-dortmund.de
%
%   Für Fragen und Anregungen zur Vorlage: info@ls7.cs.uni-dortmund.de
%
%   Stand: 06.07.2024
%
% ----------------------------------------------------------------------

\RequirePackage{ifthen}
% -----------------------------------------------------------------------------------------
% Option: deutsch/englisch
\newboolean{spracheDT}
\setboolean{spracheDT}{true} % Zuweisung auf ''false'' sofern in englischer Sprache

%
% Arbeitsbezeichnung: Bachelor-Arbeit, Master-Arbeit
%
\ifthenelse{\boolean{spracheDT}}{ % deutsch
\newcommand \Arbeitsbezeichnung{Bachelor-Arbeit}
\newcommand \Autor{Lars Naumann}
\newcommand \Arbeitstitel{Deep Learning-gestützte Ausrichtung von räumlichen Entitäten für den 5D-Druck}
\newcommand \Erstgutachter{PD Dr. Frank Weichert}
\newcommand \Zweitgutachter{Prof. Dr. Heinrich Müller}
\newcommand \ErstLehrstuhl{Lehrstuhl Informatik VII}
\newcommand \ErstLehrstuhltitel{Computergraphik}
}{ % englisch
\newcommand \Arbeitsbezeichnung{Bachelor's Thesis}
\newcommand \Autor{Lars Naumann}
\newcommand \Arbeitstitel{Deep Learning-supported Orientation of spatial Entities for 5D-Printing}
\newcommand \Erstgutachter{PD Dr. Frank Weichert}
\newcommand \Zweitgutachter{Prof. Dr. Heinrich Müller}
\newcommand \ErstLehrstuhl{Computer Science VII}
\newcommand \ErstLehrstuhltitel{Computer Graphics}
}

% -----------------------------------------------------------------------------------------
% Option: Zweiter Lehrstuhl
\newboolean{boolkeinZweitLS}
\setboolean{boolkeinZweitLS}{true} % Zuweisung auf ''false'' sofern zweiter Lehrstuhl beteiligt
\ifthenelse{\boolean{boolkeinZweitLS}}{
\newcommand \ZweitLehrstuhl{}
\newcommand \ZweitLehrstuhltitel{}
}{
\newcommand \ZweitLehrstuhl{Lehrstuhl Informatik XII}
\newcommand \ZweitLehrstuhltitel{Eingebettete Systeme}
}

\RequirePackage{ifpdf} \ifpdf
  \pdfoutput=1
  \pdftrue
  \message{pdfLaTeX}
\ifthenelse{\boolean{spracheDT}}{ % deutsch
  \documentclass[pdftex,12pt,a4paper,twoside,ngerman]{scrbook}
}{ % englisch
  \documentclass[pdftex,12pt,a4paper,twoside,english]{scrbook}
}  
  \usepackage{float}
  \usepackage[pdftex]{thumbpdf}
  \usepackage[pdftex]{graphicx}
  \usepackage[pdftex]{hyperref}
  \usepackage{pdfpages}
  \pdfoutput=1
  \pdfcompresslevel=9
  \DeclareGraphicsExtensions{.pdf,.jpg,.png}
\else
  \pdffalse
  \message{LaTeX}
  \ifthenelse{\boolean{spracheDT}}{ % deutsch
  \documentclass[dvips,12pt,a4paper,twoside,ngerman]{scrbook}
}{ % englisch
  \documentclass[dvips,12pt,a4paper,twoside,english]{scrbook}
}  
  \usepackage{float}
  \usepackage{graphicx}
  \usepackage{epsf}
  \usepackage[dvips]{hyperref}
  \DeclareGraphicsExtensions{.eps}
\fi


% Informationen fuer pdf-File festlegen
\hypersetup
{
    pdfauthor = {\Autor},
    pdftitle = {\Arbeitstitel},
    pdfsubject = {\Arbeitsbezeichnung, TU Dortmund, Fakult{\"a}t f{\"u}r Informatik},
    pdfproducer = {LaTeX},
    pdfview = FitV,
    pdfstartview = FitV,
    pdfhighlight = /I,
    pdfborder = 0 0 0,
    colorlinks = false,
    bookmarksopen,
    bookmarksopenlevel = 1,
    bookmarksnumbered = false,
    plainpages = false
}%


% Seitenformat anpassen
\usepackage[a4paper,left=3.5cm,right=2.5cm,bottom=3.5cm,top=3cm]{geometry}
\setlength{\headheight}{15pt}
% -------------------------------------------------------------------
% Grafikpakete einbinden
\usepackage{amsmath,amssymb}
\usepackage{flafter}
\usepackage{subfigure}
\usepackage{tikz}
\usepackage{tikz-3dplot}

% -------------------------------------------------------------------
\usepackage{ifthen}
\usepackage{listofitems}

% -------------------------------------------------------------------
\usepackage[absolute,overlay]{textpos}
\setlength{\TPHorizModule}{1mm}
\setlength{\TPVertModule}{\TPHorizModule}
\textblockorigin{0mm}{0mm}
\usepackage{fix-cm}
\usepackage{setspace}
\usepackage{scrhack}
% -------------------------------------------------------------------
% Korrekte Darstellung der Umlaute

\ifthenelse{\boolean{spracheDT}}{ % deutsch
  \usepackage[german,ngerman]{babel}
  \usepackage[autostyle=true,german=quotes]{csquotes}
}{ % englisch
  \usepackage[english]{babel}
  \usepackage[autostyle=true]{csquotes}
}

\usepackage[utf8]{inputenc}
\usepackage[T1]{fontenc}
\usepackage{ae,aecompl}


% -------------------------------------------------------------------
\usepackage[backend=biber, style=alphabetic, doi=false, isbn=false, url=false,  maxnames=3, minnames=1, maxbibnames=99, minbibnames=99]{biblatex}
\addbibresource{Literatur.bib} % Dateiname der Literaturdatenbank(en)

% -------------------------------------------------------------------
% URLs
\usepackage{url}

% -------------------------------------------------------------------
% Caption anpassen
\usepackage[margin=0pt,font=small,labelfont=bf]{caption}

% -------------------------------------------------------------------
% Erweitere Tabellen
\usepackage{booktabs}

% -------------------------------------------------------------------
% Eurosymbol
\usepackage{eurosym}

% -------------------------------------------------------------------
% Zeilenabstand einstellen
\renewcommand{\baselinestretch}{1.25}
% Floating-Umgebungen anpassen
\renewcommand{\topfraction}{0.9}
\renewcommand{\bottomfraction}{0.8}

% -------------------------------------------------------------------
% Keine einzelnen Zeilen beim Anfang eines Abschnitts (Schusterjungen)
%\clubpenalty = 10000
% Keine einzelnen Zeilen am Ende eines Abschnitts (Hurenkinder)
%\widowpenalty = 10000 \displaywidowpenalty = 10000

\parindent=0cm


% -------------------------------------------------------------------
% Kopfzeile hinzufuegen
\usepackage{fancyhdr}
\usepackage{extramarks}

\pagestyle{fancy}
\renewcommand{\chaptermark}[1]{\markboth{#1}{}}
\renewcommand{\sectionmark}[1]{\markright{#1}{}}

\fancyhf{}
\fancyhead[LE,RO]{\thepage}
\fancyhead[RE]{\textit{\nouppercase{\leftmark}}}
\fancyhead[LO]{\textit{\nouppercase{\rightmark}}}

\fancypagestyle{plain}{ %
\fancyhf{} % remove everything
\renewcommand{\headrulewidth}{0pt} % remove lines as well
\renewcommand{\footrulewidth}{0pt}} \pagestyle{headings}

\newcommand*{\quelle}{%
  \footnotesize Quelle:
}

% -------------------------------------------------------------------
% Eigene Farben definieren
\usepackage{color}
\definecolor{TUGreen}{rgb}{0.517,0.721,0.094}
\definecolor{TUOrange}{rgb}{1.0,0.7176,0.0}
\definecolor{BrightGray}{gray}{0.9}
\definecolor{DarkGray}{gray}{0.2}
\definecolor{white}{rgb}{1,1,1}
\definecolor{black}{rgb}{0,0,0}
\definecolor{red}{rgb}{1,0,0}




% -------------------------------------------------------------------
% Programm-Listings einbinden und formatieren
\usepackage{listings}

\lstdefinestyle{C++}
{
language=C++,
backgroundcolor=\color{BrightGray},
keywordstyle=\texttt\bfseries,  %\color{TUGreen}\bfseries,
commentstyle=\color{DarkGray},
stringstyle=\color{red},
showstringspaces=false,
basicstyle=\small\color{black},
numbers=left,
captionpos=b,
tabsize=4,
breaklines=true
}


% -------------------------------------------------------------------
% Algorithmen
\usepackage[plain,chapter]{algorithm}
\usepackage{algorithmic}

\usepackage{enumerate}

% -------------------------------------------------------------------
% Algorithmen anpassen
\ifthenelse{\boolean{spracheDT}}{ % deutsch
\renewcommand{\algorithmicrequire}{\textit{Eingabe:}}
\renewcommand{\algorithmicensure}{\textit{Ausgabe:}}
\floatname{algorithm}{Algorithmus}
\renewcommand{\listalgorithmname}{Algorithmenverzeichnis}
\renewcommand{\algorithmiccomment}[1]{\color{grau}{// #1}}
}{ % englisch
\renewcommand{\algorithmicrequire}{\textit{Input:}}
\renewcommand{\algorithmicensure}{\textit{Output:}}
\floatname{algorithm}{Algorithm}
\renewcommand{\listalgorithmname}{List of Algorithms}
\renewcommand{\algorithmiccomment}[1]{\color{grau}{// #1}}
}



% -------------------------------------------------------------------
% -------------------------------------------------------------------
% -------------------------------------------------------------------
\begin{document}


% Titelseite ---------------------------------------------------------
%
\pdfbookmark{Titelseite}{pdf:title}
\pagenumbering{alph}
\pagestyle{empty}
\include{kapitel/titelseite}

\pagestyle{empty} \cleardoublepage

% Inhaltsverzeichnis -------------------------------------------------
%
\pdfbookmark{Inhaltsverzeichnis}{pdf:toc}
\tableofcontents

\cleardoublepage \pagestyle{headings}

% Mathematische Notation -----------------------------------------------
%
\pagestyle{empty}
\pdfbookmark{Mathematische Notation}{pdf:Notation}
\include{kapitel/notation}
\cleardoublepage

\pagenumbering{arabic}
\pagestyle{fancy}

% Inhalte --------------------------------------------------------------
%
%========================================================================================
% TU Dortmund, Informatik Lehrstuhl VII
%========================================================================================

\chapter{Einleitung}
\label{Einleitung}

\section{Motivation und Hintergrund}
\label{Motivation_und_Hintergrund}

Die additive Fertigung ist ein Feld mit viel Potenzial. Es ist eine Technologie, die in der Fertigung immer wichtiger wird. Schnelle Produktion von Prototypen und Ersatzteilen sind nur einige der Anwendungsbereiche von dem 3d-Druck \cite{fastermann20123d}.
Allerdings gibt es noch Probleme, zum Beispiel mit der strukturellen Integrität von Bauteilen oder der Materialverschwendung durch Stützstrukturen. Der 5d-Druck biete eine Möglichkeit beide Probleme zu adressieren, 
indem das Bauteil während des Drucks gedreht wird. Dadurch ist es im 5d-Druck möglich Ausrichtungen für einzelnen Flächen eines Bauteils zu wählen, indem die Platform auf der das Bauteil platziert wird, sich in zwei Achsen kippen kann.
Um eine optimale Ausrichtung zu finden, gibt es bis jetzt noch keine weit anerkannte automatische Lösung. Eine Möglichkeit ist die Verwendung von neuronalen Netzen, um eine optimale Ausrichtung zu schätzen.
Die Abbildung \ref{Beispiel für einen 5d-Drucker} zeigt einen 5d-Drucker. Es lässt sich erkenne, dass die Platform gekippt ist.

\begin{figure}[t]
    \centering
    \includegraphics[scale=0.3]{bilder/5d-Drucker3.jpg}
    \caption[Beispiel für einen 5d-Drucker]{Beispiel für einen 5d-Drucker}
            \label{Beispiel für einen 5d-Drucker}
    \quelle\url{https://i.ytimg.com/vi/B9sdrezl6AU/maxresdefault.jpg}
\end{figure}

Neurale Netze sind gut geeignet, Muster in Daten zu erkennen. Die Idee ist es die Informations-Verarbeitungsfähigkeit des menschlichen Gehirns zu auf einem Computer zu simulieren \cite{Wang2003}. 
Das Gebiet der künstlichen Neuralen Netze hat sich bereits in vielen Anwendungsfällen, als verlässlich erwiesen. In der Bildverarbeitung werden es zum Beispiel verwendet, um Objekte in Bildern zu erkennen \cite{ketkar2021convolutional}.

\smallskip

Es ist daher naheliegend, dass sie auch verwendet werden können, um Muster in einer Menge an Raumpunkten zu erkennen. Eines Neurales Netz kann gut über Daten abstrahieren und auch für unbekannte komplexe Daten eine gute Lösung finden, wo klassische Algorithmen versagen könnten.
Auch für die Ausrichtung von Bauteilen haben sich neurale Netze bereits als gute Lösung erwiesen \cite{zelder2025learning}. Es wurde mit der PointNet Architektur \cite{qi2017pointnet} gezeigt, 
dass Neurale Netze in der Lage seinen können eine Ausrichtung für einfache geometrische Strukturen zu finden. Wenn man diese Idee weiterentwickelt könnte sie eine genutzt werden, um Ausrichtungen für die Flächen eines Bauteils im 5d-Druck zu bestimmen.
Zu diesem Zweck ist es notwendig lokale Merkmale zu erkennen, was die Nutzung von PointNet++ erfordert \cite{PointNet++}. Es ist im Gegensatz zu Pointnet in der Lage lokale Merkmale zu erkennen.

\section{Aufbau der Arbeit}
\label{Aufbau_der_Arbeit}

In dieser Arbeit wird versucht das Problem der Ausrichtung von Bauteilen im 5d-Druck durch die Verwendung von tiefen neuralen Netzen zu lösen. Begonnen wird mit der Aufteilung eines Bauteils in kohärente Flächen, 
wofür mehrere klassische Clustering verfahren genutzt werden. Anschließend wird ein Neurales Netz trainiert, dass die Flächen eines Bauteils auf Rotationen abbildet, die das Bauteil optimal für den 5d-Druck ausrichten.
Für das Neurale Netz wird eine angepasste Version von einem Bereits existierenden Neuralem Netz für die Merkmalsextraktion verwendet \cite{PointNet++}. Zum Schluss wird die Leistung des neuronalen Netzes evaluiert, anhand der Ausrichtung auf unterschiedlichen Objekten
und der Widerstandsfähigkeit gegen Störungen. 



\cleardoublepage
%========================================================================================
% TU Dortmund, Informatik Lehrstuhl VII
%========================================================================================

\chapter{Topologieaspekte in der Additive Fertigung}
\label{Topologieaspekte in der Additive Fertigung}

\section{3d-Druck}
\label{3d-Druck}

Der 3d-Druck ist eine Methode der Fertigung. Es wird auch additive Fertigung genannt, da das Bauteil durch das Hinzufügen von Material Schicht für Schicht erstellt wird \cite{standardization2015additive}.
Um eine Bauteil 3d-Drucken zu können, wird das Bauteil im Computer in mehrere Schichten aufgeteilt. Diese Schichten werden dann nacheinander gedruckt, bis das Bauteil fertig ist.
Es gibt mehrere Typen von 3d-Druck. Die Typen beinhalten: "binding jetting",
"directed energy deposition", "material extrusion", "material jetting", "powder bed fusion", "sheet lamination and vat"
"photopolymerization". Alle Methoden tragen Material auf. Die am weitesten verbreitete Methode ist "Materials extrusion" bei der meistens Plastik erhitzt wird und dann das geschmolzene Plastik aufgetragen wird.
Die Materialien die genutzt werden können sind aber mehr als Plastik und beinhalten unter anderem noch Metall, Polymere, Komposit Materialien und noch mehr. Metalle werden in der Luft und Raumfahrt verwendet, da sie sehr gute physische Eigenschaften haben.
Das benutzte Metall kommt auf den genauen Anwendungsbereich an. Plastik hat den Vorteil, das es leicht zu erhitzten und billig ist. Es wird unter anderem für die Produktion von Prototypen benutzt, kann aber auch für fertige Produkte verwendet werden.
Komposit Materialien, wie Glasfaser, dagegen sind anpassungsfähig und leicht. Sie werden ebenfalls in der Luft und Raumfahrt eingesetzt. Der 3d-Druck ist eine Technologie mit vielen Anwendungsmethoden und Bereichen \cite{shahrubudin2019overview}. 

\begin{figure}[t]
    \centering
    \includegraphics[scale=0.8]{bilder/3d-Drucker.jpg}
    \caption[Schematische Darstellung eines 3d-Druckers]{Schematische Darstellung eines 3d-Druckers}
            \label{Schematische Darstellung eines 3d-Druckers}
    \quelle\url{https://www.linux-magazin.de/wp-content/uploads/2020/11/b01.jpg}
\end{figure}

In der Abbildung \ref{Schematische Darstellung eines 3d-Druckers} ist eine schematische Darstellung eines 3d-Druckers zu sehen. Dabei ist oben die Düse, welche das Material erst schmilzt und dann aufträgt. Die Düse ist lässt sich in x und y Richtung bewegen 
und kann so jede beliebige Position auf der Plattform erreichen. Unten kann man die Platform sehen, auf der das Bauteil gedruckt wird. 
Sie kann sich nach unten bewegen, damit Platz für die nächste Schicht geschaffen wird. Die Funktion des Stützmaterials und der Stützschichten wird in Abschnitt \ref{Überhang-Problem} erklärt.
Der in Abbildung \ref{Schematische Darstellung eines 3d-Druckers} zusehen 3d Drucker ist ein Beispiel für den Aufbau eines 3d-Druckers. Es gibt 3d Drucker, die anders funktionieren, aber das Grundprinzip ist bei allen gleich.
Alle haben eine Düse, die das Material ausgibt und diese Düse bewegt sich, um Schichten abzuarbeiten.

\section{Überhang-Problem}
\label{Überhang-Problem}

Im 3d-Druck werden Bauteile traditionell Schicht für Schicht gebaut. Das Bauteil wird vor dem Druck in dünne horizontale Schichten zerlegt. Jede Schicht wird dann nacheinander gedruckt. 
Dabei kann es zu Problemen kommen, wenn eine Schicht über eine vorherige Schicht hinausragt, ohne dass darunter Material ist. Dies wird als Überhang-Problem bezeichnet. Um diesem Problem entgegenzuwirken werden Stützstrukturen eingesetzt.


\begin{figure}[t]
    \centering
    \includegraphics[scale=0.5]{bilder/Stützstruktur.jpeg}
    \caption[Vorschau eines Bauteils mit Stützstrukturen]{Vorschau eines Bauteils mit Stützstrukturen}
            \label{Vorschau eines Bauteils mit Stützstrukturen}
    \quelle\url{https://i.stack.imgur.com/cIDB3.png}
\end{figure}

In der Abbildung \ref{Vorschau eines Bauteils mit Stützstrukturen} ist ein Bauteil zu sehen, das Stützstrukturen benötigt. Stützstrukturen erhöhen den Materialaufwand, wie man in Abbildung \ref{Vorschau eines Bauteils mit Stützstrukturen} sehen kann.
Die Strukturen werden nach dem Druck entfernt, was das zusätzlich benötigte Material zu Müll mach. Durch das Hinzufügen von Stützstrukturen verlängert sich auch die Druckzeit. Dieses Problem kann durch die richtige Ausrichtung des Bauteils minimiert werden, 
aber bei komplexen Bauteilen ist es unwahrscheinlich, dass eine Ausrichtung gefunden wird, die keine oder wenige Stützstrukturen benötigt. Stützstrukturen sind nicht der einzige Weg das Übergangsproblem zu bekämpfen. Eine Anpassung der Druckgeschwindigkeit 
und der Materialtemperatur können die Notwendigkeit von Stützstrukturen reduzieren. Dabei ist ein Überhang von 40 Grad möglich ohne Stützstrukturen oder Verlust an Qualität möglich \cite{jiang2018investigation}. 

\section{Fertigungs-Geschwindigkeit}
\label{Fertigungs-Geschwindigkeit}

Wie bei jeder Art der Fertigung ist die Produktionsgeschwindigkeit ein wichtiger Faktor. In der additiven Fertigung wird die Produktionsgeschwindigkeit durch verschiedene Faktoren beeinflusst.
Eine dieser Faktoren wurde in Abschnitt \ref{Überhang-Problem} beschrieben. Das Hinzufügen von Stützstrukturen erhöht den Drückzeit, da mehr Material gedruckt werden muss. Das ist ein weiterer Grund warum es wichtig ist Stützstrukturen zu minimieren.
Dementsprechend ist die Ausrichtung eines Bauteils wichtig, da eine gute Ausrichtung die benötigten Stützstrukturen minimieren kann.

\smallskip

Der offensichtlichste Faktor ist die Größe des Bauteils und die Druckgeschwindigkeit des Druckers. Ein größeres Bauteil benötigt mehr Zeit, wenn es mit der gleichen Geschwindigkeit gedruckt wird. Die Druckgeschwindigkeit ist abhängig vom Drucker.
Unabhängig von dem Drucker gibt es physikalische Grenzen für die Druckgeschwindigkeit. Wenn eine bestimmte Geschwindigkeit überschritten wird, wird es zu Problemen mit der Genauigkeit und Qualität des Drucks kommen. 
Die Geschwindigkeit, mit der eine Düse Material ausgeben kann, ist ebenfalls ein limitierender Faktor.

\begin{figure}[t]
    \centering
    \includegraphics[scale=1]{bilder/Schichtenhöhe.jpg}
    \caption[Darstellung der Unterschiede im Bauteil abhängig von der Höhe der Schichten]{Darstellung der Unterschiede im Bauteil abhängig von der Höhe der Schichten}
            \label{Darstellung der Unterschiede im Bauteil abhängig von der Höhe der Schichten}
    \quelle\url{https://encrypted-tbn0.gstatic.com/images?q=tbn:ANd9GcS6yHOdjlgO6mhqwt_JJ22BE7VaI-OV4U4SWg&s}
\end{figure}

Ein weiterer Faktor ist die Höhe der einzelnen Schichten. In der Abbildung \ref{Darstellung der Unterschiede im Bauteil abhängig von der Höhe der Schichten} ist der Unterschied, den die Höhe der Schichten auf die Qualität des Bauteils hat, zu sehen. 
Umso dünner die Schicht umso genauer wird das Bauteil gedruckt. Wenn die Dicke der Schichten verringert wird, werden mehr Schichten benötigt, was die Strecke, die die Düse abarbeiten muss, erhöht. Dadurch verlängert sich die Druckzeit. 

\section{5d-Druck}
\label{5d-Druck}

Der 5d-Druck ist eine Erweiterung des 3d-Drucks. Ein 3d-Drucker kann die Düse in der x und y Richtung und die Platform in der z Richtung bewegen. Ein 5d-Drucker kann zusätzlich die Platform in zwei Dimensionen kippen.

\begin{figure}[t]
    \centering
    \includegraphics[scale=1.2]{bilder/5d-Drucker.jpg}
    \caption[Darstellung der beweglichen Plattform eines 5d-Druckers]{Darstellung der beweglichen Plattform eines 5d-Druckers}
            \label{Darstellung der beweglichen Plattform eines 5d-Druckers}
    \quelle\url{https://www.3dnatives.com/de/wp-content/uploads/sites/3/150217_Zurich1.jpg}
\end{figure}

In der Abbildung \ref{Darstellung der beweglichen Plattform eines 5d-Druckers} ist eine Düse und eine Plattform zu sehen. Die Platform ist sichtlich gekippt. Die Fähigkeit die Platform zu kippen ermöglicht es Bauteile zu drucken, ohne Stützstrukturen zu benötigen, 
da wenn eine Schicht am Überhängen ist, das Bauteil gedreht werden kann, sodass die nächste Schicht nicht mehr über die aktuelle Schicht überhängt. Durch das Kippen der Plattform ist es ebenfalls möglich Schichten, die nicht parallel zu der Plattform sind zu drucken.
Die bessere Anpassung an die Geometrie der Bauteile verbessert die Festigkeit des Bauteils.

Diese Fähigkeit Schichten zu erstellen, die nicht parallel zu der Plattform sind, hat den Nachteil, dass die Berechnung der Schichten komplexer wird. Die meisten Programme zu der Erstellung von Schichten unterstützen keinen 5d-Druck. 
Herausforderung bei der Erstellung von Schichten für den 5d-Druck sind unter anderem die Kollisionsvermeidung und die Berechnung der optimalen Kippwinkel für jede Schicht.

Die Kollisionsvermeidung ist ein Problem, das durch den 5d-Druck entsteht. Wenn das Bauteil sich während des Druckes drehen kann, besteht die Möglichkeit, dass die Düse mit dem Bauteil kollidiert. 
Bei dem 3d-Druck ist das nicht möglich, da die Düse immer über dem Druckbereich ist.

Optimale Kippwinkel für jede Schicht zu bestimmen benötigt die Berechnung des optimalen Winkels für jede Schicht. Dieser Prozess wird schwerer gemacht, da berücksichtigt werden muss, 
dass eine Fläche unter Umständen nicht von einem bestimmten Winkel aus erreichbar sein könnte, weil dort bereits etwas anderes gedruckt wurde, was mit der Düse kollidieren würde. Es muss also auch bei der Ausrichtung des Bauteils Kollisionsvermeidung berücksichtigt werden.

\section{Stand der Technik}
\label{Stand der Technik}

\cleardoublepage
%========================================================================================
% TU Dortmund, Informatik Lehrstuhl VII
%========================================================================================

\chapter{Flächensuche in  Bauteilen}
\label{Flächensuche in  Bauteilen}

\section{Kriterien für geeignete Flächen}
\label{Kriterien für geeignete Flächen}

Bevor ein Bauteil ausgerichtet werden, kann, müssen zuerst kohärente zusammenhängende Flächen im Bauteil identifiziert werden. Ein Bauteil ist ein Objekt, dass gedruckt werden könnte. Die meisten Bauteile in dieser Arbeit sind nur für die Weiterverarbeitung 
oder zum Testen da. Die Verarbeitung dieser Flächen wird in Kapitel \ref{Punktwolkenverarbeitung zur Bauteilausrichtung} behandelt.
Eine kohärente Fläche ist eine stetige Fläche. Sie hat weder Sprünge noch Lücken oder Risse. Diese Eigenschaft ist hilfreich, da eine Fläche die nicht kohärent ist definitionsgemäß an mindestens einer Stelle einen Sprung oder Lücke haben muss. 
Es ist einfacher, wenn diese Fläche als zwei oder mehr Flächen weiterverarbeitet wird, anstelle von einer Fläche die an mindestens einer Stelle getrennt ist. Eine zusammenhängende Fläche ist in dieser Arbeit eine Fläche, 
die entweder keine oder eine geringe Krümmung aufweist. Eine Krümmung kann nur bis zu einem bestimmten Punkt akzeptiert und ein Knick innerhalb einer Fläche sollte immer vermieden werden. Wie der Punkt bis, zu dem eine Krümmung akzeptabel ist festgelegt wird, 
in dem folgenden Abschnitt \ref{Clustering nach Normalvektoren in Bauteilen} erklärt. Eine Fläche mit einer Krümmung kann eine gute Ausrichtung unmöglich machen. 
Das Ziel der Ausrichtungen ist es eine Fläche senkrecht in Relation zu der Druckerplatte zu Orientieren. Bei einer gekrümmten Fläche ist dies unmöglich, da wenn man zwei Tangenten an zwei zufälligen Punkten einer gekrümmten Fläche nimmt, 
werden sie nicht parallel sein. Die Tangenten haben die Steigung der Fläche an den Punkten. Die Abbildung \ref{Darstellung von Tangenten in einer gekrümmten Fläche} zeigt eine gekrümmte Fläche in Blau und auf der Fläche zwei Tangenten, $t_1$ und $t_2$, 
die in Magenta markiert sind. Wenn die y-Achse die Grundfläche eines Druckers ist, dann lässt es sich leicht erkennen, dass die Fläche unmöglich Senkrecht zu der Druckerfläche seien kann. Sobald in einer Fläche zwei Tangenten, 
die nicht parallel zueinander sind, existieren, ist eine perfekt senkrechte Ausrichtung unmöglich. Eine Krümmung sollte dementsprechend so gering wie möglich sein. 

\begin{figure}[t]
        \centering
        \begin{tikzpicture}[scale=6]
            \draw[thick,->] (0,0) -- (2,0) node[anchor=north east]{$x$};
            \def\x{.5}
            \draw[thick,->] (0,0) -- (0,1) node[anchor=north west]{$y$};
            \draw[blue, very thick] (0,0) to [out=45,in=135] (2,0);
            \draw[magenta, very thick,->] (0.25,0.225) -- (0.75,0.425) node[anchor=south]{$t_1$};
            \draw[magenta, very thick,->] (1.25,0.425) -- (1.75,0.225) node[anchor=south]{$t_2$};
            \draw (1,0.41) circle[radius=0.5pt];
            \fill (1,0.41) circle[radius=0.5pt];
        \end{tikzpicture}
        \caption[Darstellung von Tangenten in einer gekrümmten Fläche]{Darstellung von Tangenten in einer gekrümmten Fläche}
            \label{Darstellung von Tangenten in einer gekrümmten Fläche}
    \end{figure}


Dieses Kapitel erklärt, wie in einem Bauteil kohärente und zusammenhängende Flächen gefunden werden. Dafür wird ein Bauteil durch zufällig auf der Oberfläche verteilte Punkte $p \in \mathbb{R}^{3}$ dargestellt. 
Eine Menge $\mathbb{R}^{3xN}$ dieser Punkte wird Punktwolke genannt.
Es werden in dieser Arbeit mehrere tausend Punkte benutzt, um ein Bauteil mit einer Punktwolke darzustellen. 
Danach wird die Punktwolke in mehrere kleinere Gruppen aufgeteilt.

\smallskip

Eine Aufteilung von Datenpunkten in Gruppen wird Clustering genannt, eine Gruppe ist ein Cluster. Die Aufteilung in Gruppen basiert auf den Werten der Datenpunkte, Daten mit ähnlichen Werten kommen generell in dieselbe Gruppe. Die Gruppen werden auch Cluster genannt. 
Was Ähnlichkeit bedeutet ist unterschiedlich, aber ein häufiges Ähnlichkeitsmaß und das Ähnlichkeitsmaß, welches in dieser Arbeit benutzt wird, ist die euklidische Distanz. Euklidische Distanz ist die Distanz zweier Punkt in einer Ebene oder in einem Raum.
Die Distanz, die sie beschreibt, ist die Länge der minimalen Strecke zwischen zwei Punkten. Wenn $p = (p_1, p_2, ... , p_i)$ und $q = ((q_1, q_2, ... , q_i))$ Koordinaten in einem i-Dimensionalen Raum sind, dann ist die euklidische Distanz gleich
$\sqrt{\sum_{i=1}^{n}(q_i - p_i)^2}$.

\begin{figure}[t]
    \centering
    \includegraphics[scale=0.5]{bilder/Clustering.png}
    \caption[Clustering von verstreuten zweidimensionalen Punkten]{Clustering von verstreuten zweidimensionalen Punkten}
            \label{Clustering von verstreuten zweidimensionalen Punkten}
    \quelle\url{https://i.stack.imgur.com/cIDB3.png}
\end{figure}


In der Abbildung \ref{Clustering von verstreuten zweidimensionalen Punkten} sind zwei zweidimensionale Punktwolken. Die linke ist nicht geclustert, die rechte ist geclustert, beide stellen dieselbe Punktwolke dar.
Die rechte Punktwolke ist mit drei Farben markiert, Punkte derselben Farbe sind im selben Cluster. In der rechten Punktwolke ist das Ergebnis von einem Clustering verfahren dargestellt. 
Clustering ist ein verfahren, mit dem ein Satz an Daten in Gruppen eingeordnet wird. Die Elemente einer Gruppe sind sich innerhalb ähnlicher als Elemente aus anderen Gruppen. Es lässt sich auf der rechten Seite der Abbildung \ref{Clustering von verstreuten zweidimensionalen Punkten}
erkennen, dass alle Punkte einer Farbe nahe beieinander sind. Außerdem kann man eine klare Trennung zwischen den Farben erkennen. Es gibt keine Ausnahmen, wo ein Punkt der einen Farbe in dem Bereich einer anderen Farbe ist. Es lässt keine Ausnahmen zu.
Clustering wird benutzt, um Konzentrationen von ähnlichen Punkten zu finden.

\section{Clustering nach Normalvektoren in Bauteilen}
\label{Clustering nach Normalvektoren in Bauteilen}

\subsection{Normalvektoren}
\label{Normalvektoren}

Um zu Messen, ob zwei Punkte Teil derselben Fläche sein sollen, muss die Ausrichtung eines Punktes gemessen werden. Punkte mit einer ähnlichen Ausrichtung gehören zu derselben Fläche. Mit der Ausrichtung eines Punktes ist ein Vektor, 
der, wenn man eine Ebene, die genau die Steigung des Punktes hat und so tangential zu ihm ist nimmt, wäre die Ausrichtung des Punktes Senkrecht zu dieser Ebene. Solche Vektoren werden Normalvektor genannt. Es sind nur die Normalvektoren relevant, 
die einen dazugehörigen Punkt in der Punktwolke des Bauteils haben.

    \begin{figure}[t]
        \centering
        \begin{tikzpicture}[scale=6]
            \draw[thick,->] (0,0) -- (2,0) node[anchor=north east]{$x$};
            \def\x{.5}
            \draw[thick,->] (0,0) -- (0,1) node[anchor=north west]{$y$};
            \draw[blue, very thick] (0,0) to [out=45,in=135] (2,0);
            \draw[red, very thick,->] (1,0.41) -- (1,1) node[anchor=north west]{$n_i$};
            \draw[ultra thick] (1,0.41) node[anchor=north west]{$p_i$};
            \draw (1,0.41) circle[radius=0.5pt];
            \fill (1,0.41) circle[radius=0.5pt];
        \end{tikzpicture}
        \caption[Darstellung von Normalvektoren an einem Punkt der Bauteiloberfläche]{Darstellung von Normalvektoren an einem Punkt der Bauteiloberfläche}
            \label{Darstellung von Normalvektoren an einem Punkt der Bauteiloberfläche}
    \end{figure}

In der Abbildung \ref{Darstellung von Normalvektoren an einem Punkt der Bauteiloberfläche} ist die Bauteiloberfläche blau markiert. Der Punkt $p_i$ ist ein beliebiger Punkt auf der Bauteiloberfläche. Der Normalvektor $n_i$ geht von dem Punkt $p_i$ aus senkrecht nach oben.
Er ist im 90 Grad Winkel zu der Bauteiloberfläche. Die Normalvektoren in der Abbildung \ref{Darstellung von Normalvektoren an einem Punkt der Bauteiloberfläche} sind zweidimensionale, im Gegensatz dazu sind die Normalvektoren auf den Punkten im Bauteil dreidimensional.
Die Normalvektoren werden für jeden Punkt anhand der Oberfläche des Bauteils bestimmt. 

\smallskip

Eine Aufteilung der Normalvektoren in Cluster ist notwendig, um Flächen zu erkennen. Dafür werden die Normalvektoren als dreidimensionale Punkte dargestellt, die euklidische Distanz zwischen den Normalvektoren gib ihre Ähnlichkeit an. 
Punkte mit ähnlichem Normalvektor haben eine ähnliche Ausrichtung uns sollte teil derselben Fläche sein. Die Darstellung als dreidimensionale Punkte erlaubt funktioniert nur, wenn alle Normalvektoren normiert sind. Ein normierter Vektor hat eine Länge von 1. 

\subsection{Hierarchisches Clustering}
\label{Hierarchisches Clustering}

Viele Methoden des Clusterings benötigen eine vorgegebene Anzahl an Gruppen, in die sie die Punkte enteilen sollen. Es muss davon ausgegangen werden, dass bei einem Bauteil die Anzahl der Flächen die gefunden werden können unbekannt ist.
Das macht Methoden, die eine Anzahl an Gruppen benötigen ungeeignet. Stattdessen wird eine Methode genutzt, die mit allen Datenpunkten zum Start Ihre eigene Gruppe haben, sodass es für jeden Datenpunkt eine Gruppe gibt. 
Die ähnlichsten zwei Gruppen werden dann zusammengefügt.
Auf diesen Gruppen wird der Vorgang wiederholt, bis das Zusammenfügen von Gruppen immer einen vor der Ausführung festgelegten Wert an Differenz überschreitet. Es passt so die Anzahl der Gruppen dem Bauteil an. 
Dieses Vorgehen wird Hierarchie Clustering genannt.


\begin{figure}[t]
    \centering
    \includegraphics[scale=0.6]{bilder/Hierachie.png}
    \caption[Beispiel von Hierarchie Clustering als Dendrogramm]{Beispiel von Hierarchie Clustering als Dendrogramm}
            \label{Beispiel von Hierarchie Clustering als Dendrogramm}
\end{figure}

Die Abbildung \ref{Beispiel von Hierarchie Clustering als Dendrogramm} stellt das Vorgehen von Hierarchie Clustering dar. Dabei ist die y-Achse der Distanzwert und auf der x-Achse sind die Cluster markiert. Das Clustering fängt unten, bei $y = 0$ an. Dort ist jeder Punkt ein Cluster. 
Danach wird die Distanz zwischen allen Clustern gemessen und die zwei ähnlichen Cluster werden zusammen genommen zu einem Cluster. Dieses Vorgehen wird so lange wiederholt, bis ein bestimmter Distanzwert zwischen allen Clustern überschritten wurde. 
In der Abbildung \ref{Beispiel von Hierarchie Clustering als Dendrogramm} kann man das Zusammenfügen von Clustern sehen, wenn zwei über eine senkrechte Linie verbunden werden. Aus der Linie, welche die beiden Cluster verbindet, kommt eine weitere Linie, 
die das neue zusammengefügte Cluster symbolisiert.
Das hat den Vorteil, dass die Anzahl der Cluster in einem Bauteil von dem Bauteil selber abhängt. Es ist nicht nötig eine feste Anzahl an Clustern festzulegen, sondern nur eine Ähnlichkeit, welche die Cluster haben sollen.
In der Abbildung \ref{Beispiel von Hierarchie Clustering als Dendrogramm} stoppt der Algorithmus nicht, nachdem alle Cluster einen vorbestimmten Distanzwert überschreiten, sondern fügt so lange Cluster zusammen, bis nur noch eins übrig ist.

\smallskip

Das Zusammenfügen von Gruppen basiert auf Ähnlichkeit der Gruppen. In jedem Schritt werden die ähnlichsten Gruppen zusammengefasst. Die Ähnlichkeit zwischen Gruppen wird über die Ward Methode bestimmt \cite{Ward}. Diese Methode vereinigt die zwei Gruppen,
mit dem geringsten Anstieg an Varianz. Dafür wird für jede Gruppe $G$ die Varianz mittels $\sum_{i \in G}(||\overline{x} - x_i||^2)$ bestimmt. $\overline{x}$ ist dabei der Durchschnitt aller Punkte in G. Der Veränderung der Varianz bei dem Zusammengefügen
zweier Gruppen $A,B$ kann durch $\sum_{i \in A \bigcup B}(||\overline{x} - x_i||^2) - \sum_{i \in G}(||\overline{x} - x_i||^2) - \sum_{i \in G}(||\overline{x} - x_i||^2)$ bestimmt werden. Von der Varianz der zusammengefügten Gruppen wird die Varianz der 
beiden einzelnen Gruppen abgezogen, das Ergebnis ist die Erhöhung der Varianz durch das Zusammenfügen von A und B. Die Gruppen zwischen denen diese Erhöhung minimal ist, werden zusammengefasst.

\smallskip

Auf diese Art und Weise werde die Cluster zwischen Normalvektoren gesucht. Alle Normalvektoren, die in einem Cluster sind, sind so ähnlich, dass das hierarchische Clustering sie nicht weiter zusammengefügt hat. 
Der Distanzwert auf der y-Achse in \ref{Beispiel von Hierarchie Clustering als Dendrogramm} dient dabei als Fehlermaß für die Flächen. Wenn das Fehlermaß null ist, werden nur Flächen gefunden, die absolut Flach sind, 
aber bei komplexen Bauteilen ist es notwendig ein höherer Fehler zuzulassen, um gebogene Flächen zu finden. Bei komplexen Bauteilen würde die Suche nach ausschließlich gerade Flächen in sehr viele extrem kleinen Flächen resultieren, 
da viele Bauteile keine gerade Flächen haben werden. Eine Abweichung zuzulassen ist also sinnvoll.


\begin{figure}[t]
    \centering
    \includegraphics[scale=0.7]{bilder/Normal_Cluster.png}
    \caption[Hierarchisches Clustering eines Bauteils nach Normalvektor]{Hierarchisches Clustering eines Bauteils nach Normalvektor}
            \label{Hierarchisches Clustering eines Bauteils nach Normalvektor}
\end{figure}

Das Clustering nach Normalvektoren ist nur geeignet um Flächen zu finden, die, innerhalb des Distanzwertes als Fehlermaß, Flach sind. Das Clustering gibt keine Garantie, dass die gefundenen Flächen kohärent sind. In Abbildung \ref{Hierarchisches Clustering eines Bauteils nach Normalvektor}
ist sichtbar, dass die in Grün markierten Flächen, parallel zueinander, aber nicht kohärent miteinander, sind. Die Flächen die in Grün markiert sind, sind Teil desselben Clusters, aber sollten es nicht sein. 
Die roten Punkte markieren das Bauteil, in dem die Flächensuche durchgeführt wurde. 

\smallskip

Der nächste Abschnitt \ref{Separation von parallelen Flächen} beschäftigt sich damit wie die in den hierarchischen Clustering gefundenen Flächen getrennt werden.

\section{Separation von parallelen Flächen}
\label{Separation von parallelen Flächen}

Alle Flächen in einem Cluster des Normalvektorclusterings aus Abschnitt \ref{Clustering nach Normalvektoren in Bauteilen} sind in etwa parallel, aber räumlich voneinander getrennt. 
Das macht sie ungeeignet für die Bildung einer Ausrichtung in Kapitel \ref{Punktwolkenverarbeitung zur Bauteilausrichtung}. In Abbildung \ref{Hierarchisches Clustering eines Bauteils nach Normalvektor} lässt sich erkenne, 
dass die Punkte, die kohärente Flächen Bilden immer sehr dicht beieinander sind. Zwischen den erkennbaren Flächen sind große Lücken ohne Punkte. Die Dichte der Punkte kann somit genutzt werden, um die einzelnen Flächen voneinander zu trennen.

\smallskip

DB-Scan ist eine Dichte basierender Clustering Methode. Es braucht wenig Informationen über die zu gruppierenden Daten, funktioniert auf Clustern jeglicher Form und ist auch auf großen Datensätzen effektiv \cite{DBScann}.

\smallskip

Die räumliche Trennung zwischen den Flächen in einem Cluster kann genutzt werden, um sie zu trennen. Dafür wird der DB-Scan Algorithmus verwendet. DB-Scan benötigt einen Radius und eine Mindestanzahl an Nachbarn. 
Der Radius wird $esp$ und die Mindestanzahl an Nachbarn wird $minPts$ in Abbildung \ref{Beispielhafte Darstellung von DB-Scan Algorithmus} genannt. Die Eingabe muss relativ zu der Dichte der Flächen bestimmt werden.

\begin{figure}[t]
    \centering
    \includegraphics[scale=0.5]{bilder/DB-scan.png}
    \caption[Beispielhafte Darstellung von DB-Scan Algorithmus]{Beispielhafte Darstellung von DB-Scan Algorithmus}
            \label{Beispielhafte Darstellung von DB-Scan Algorithmus}
    \quelle\url{https://machinelearninggeek.com/wp-content/uploads/2020/10/image-58.png}
\end{figure}


Die Abbildung \ref{Beispielhafte Darstellung von DB-Scan Algorithmus} zeigt ein Beispiel für die Funktion von dem DB-Scan Algorithmus, der im Folgenden erklärt wird. 
Der Algorithmus nimmt einen Punkt, der in keinem Cluster ist und überprüft, wie viel Punkte in einem Radius $esp$ um ihn herum sind. Wenn innerhalb des Radiuses mindestens $minPts$ viele Punkte sind, dann ist der Punkt ein Kernpunkt (engl. Core Point) eines neuen Clusters.
Alle Punkte, die in dem Radius des Punktes sind, werden Teil des Clusters und genauso überprüft, wie der Startknoten des Clusters. Dieses Vorgehen wird fortgeführt, bis keine Punkte mehr hinzugefügt werden können. Wenn ein Punkt Teil des Clusters ist, 
aber nicht mindestens $minPts$ viele Punkte in seiner Nachbarschaft hat, dann ist dieser Punkt ein Randpunkt (engl. Borderpoint), aber immer noch teil des Clusters.
Wenn weniger als $minPts$ innerhalb des Radiuses sind, dann gilt der Punkt als Noise. Das bedeutet aber nicht unbedingt, dass der Punkt Noise bleibt. Er kann auch zu einem Randpunkt eines Noch nicht existierenden Clusters werden.
Es werden so lange neue Punkte außerhalb von Clustern gewählt, bis alle Punkte Teil eines Clusters oder Noise sind.

\smallskip

Alle Flächen innerhalb eines Cluster der Normalvektoren sind räumlich getrennt. Wenn der Algorithmus auf die vorher gefunden Cluster angewandt wird, findet er kohärente Flächen als Cluster: In diesen Flächen liegen die Punkte sehr nahe beieinander, 
wodurch die Punkte in der Fläche meistens die $minPts$ Bedingung erfüllen. Andere Flächen werden wahrscheinlich nicht innerhalb von $esp$ der Randpunkte einer Fläche liegen. Es gibt keine Garantie, dass alle Flächen getrennt werden, 
aber in den meisten Fällen wird es passieren.
Ein kleinerer Radius verringert die Chance die Trennung von zwei Flächen zu übersehen, aber benötigt auch mehr Punkte, um verlässlich zu funktionieren. Mit zu wenigen Punkten könnte eine ununterbrochene Fläche fehlerhaft getrennt werden.
Die Erhöhung der Punkte führt außerdem zu einer erhöhten Rechenzeit. DB-Scan ist auch auf großen Datensätzen effizient, mit einer Zeit Komplexität Relativ zu der Eingabelänge von $O(n \log n)$ \cite{gunawan2013faster}. 
Davor muss das hierarchische Clustering aus Abschnitt \ref{Hierarchisches Clustering} ausgeführt werden. Es hat eine Zeitkomplexität von $O(n^2)$ \cite{day1984efficient}. 
Mit einer quadratischen Zeitkomplexität ist das Clustern auf den Normalvektoren der Problempunkt, durch den eine die gesamte Flächensuche eine Zeitkomplexität von $O(n^2)$ hat.

\smallskip

Zuletzt werden von den finalen Clustern alle aussortiert, die eine bestimmte Punktanzahl unterschreiten. Flächen mit zu wenigen Punkten würden schwer zu verarbeiten sein, daher ist es einfacher sie nicht zu betrachten. 
Der Verlust an Punkten dadurch wird nur gering sein. Für eine echte Teilausrichtung dürften allerdings keine Punkte verloren gehen. Jeder Punkt muss gedruckt werden. Das Problem ließe sich im Realfall lösen, indem das Aussortieren weggelassen wird.

\cleardoublepage
%========================================================================================
% TU Dortmund, Informatik Lehrstuhl VII
%========================================================================================

\chapter{Punktwolkenverarbeitung zur Bauteilausrichtung}
\label{Punktwolkenverarbeitung zur Bauteilausrichtung}

Um ein Bauteil korrekt auszurichten, muss eine Fläche des Bauteils auf eine Ausrichtung abgebildet werden. Die Flächen in den Bauteilen wurden in Kapitel \ref{Flächensuche in  Bauteilen} identifieziert und als Punktwolken gespeichert.
Punktwolken wurden im Kapitel \ref{Flächensuche in  Bauteilen} bereits beschieben. Eine Punktwolke ist eine Menge an Punkten im 3D Raum, die zusammen eine Fläche oder ein Objekt darstellen.

\smallskip

In diesem Kapitel wird dargestellt, wie man eine Punktwolke, die eine Fläche in einem Bauteil darstellt, zu einer Rotation transformieren kann. Für die Transformation wird ein Neurales Netz eingesetzt, welches im Folgenden genauer erklährt wird. 
Dieses Neurale Netz wird eine Punktwolke beliebiger größe zu der Z-Invarianz transformieren. Die Z-Invarianz wird in einem folgenden Abschnitt \ref{Konstruktion der Ausrichtung} erklährt. Aus dieser Ausgabe lässt sicht dann eine vollständige Drehung erstellen.

\section{Punktwolken Transformation}
\label{Punktwolken Transformation}

Die Transformation der Punktwolken erfolgt über ein Neurales Netz. Ein Neurales Netz kann Daten verarbeiten, indem es sie durch viele Schichten an künstlichen Neuronen durchleitet. 
Ein künstliches Neuron ist eine Funktion, die als Eingabe andere künstliche Neuronen oder echte Daten nimmt. 
Die Eingabe des Neurons besteht aus den einzelnen Eingaben ($x_{i}$) multipliziert mit den Gewichten für die einzelnen Eingaben ($w_{i}$).

\begin{figure}[t]
    \centering
    \begin{math} \sum_{ i = 1 }^{ n }{ x_{i} * w_{i} } = Netzeingabe \end{math}
\end{figure}

Diese Funktion bildet die gewichteten Eingaben dann auf eine Ausgabe ab, 
die entweder an ein weiteres Neuron übergeben wird, oder die Ausgabe des Netzes darstellt.
Die Abbildung in der Aktivierungsfunktion kann jede Funktion sein, die einen Wert x auf einen anderen Wert y abbildet, allerdings sind nicht lineare Funktionen wie zum Beispiel: $y = \max(0,x)$ (ReLU) generell besser geeignet, 
da man nur mit dieser Art von Aktivierungsfunktionen nicht lineare Zusammenhänge darstellen kann.

\begin{figure}[t]
    \centering
    \includegraphics[scale=0.25]{bilder/ArtificialNeuronModel_deutsch}
    \caption[Funktionsweise eines künstlichen Neurons]{Funktionsweise eines künstlichen Neurons}
            \label{Funktionsweise eines künstlichen Neurons}
    \quelle\url{https://upload.wikimedia.org/wikipedia/commons/7/7f/ArtificialNeuronModel_deutsch.png}
\end{figure}

Ein typisches Neuron ist in Abbildung \ref{Funktionsweise eines künstlichen Neurons} dargestellt. Durch das Anpassen von Gewichten,$w_{i}$ in der Abbilding \ref{Funktionsweise eines künstlichen Neurons}, kann ein Neuron den Einfluss von Eingabedaten verändern. 
Die optimale Belegung von Gewichten wird von dem Netz über mehrere Versuche hinweg erarbeitet. Dafür muss definiert werden, was das Neurale Netz als optimales Ergebnis ansieht. Für die Anpassung der Gewichte in jedem Neuron wird ein Datensatz genutzt, 
den das Neurale Netz mehrmals durchläuft. Diese Daten werden Trainingsdaten genannt. Bei jeden Durchlauf der Trainingsdaten
übeprüft das Neurale Netz, wie gut seine Ausgaben im Vergleich zu dem vordefinierten optimalen Ergebnis ist. Dis Trainigsdaten brauchen für jeden Datenpunkt ein richtiges Ergebnis, damit das Neurale Netz sein Ergebniss mit dem vorher berechnet 
oder festgelegtem Optimum vergleichen kann. In dem Falle ist das Optimum der Normalvector zu der jeweiligen Punktwolke. 

\smallskip

Die Gewichte in den Neuronen werden vor der Anpassung zufällig festgelegt. In jeder Iteration werden die Gewichte an den Neuronen angepasst, falls dies zu einem besseren Ergebnis führt. Dafür wird über gradienten berechnet, wie sehr jedes gewicht zu dem Fehler beigetragen hat.
Über viele iterationen versucht das Netz dadruch den Fehler (engl. Loss) zu minimieren.

\begin{figure}[t]
    \centering
    \includegraphics[scale=0.4]{bilder/Regression Net.jpg}
    \caption[Schematische Darstellung von Regressionsnetz]{Schematische Darstellung von Regressionsnetz}
            \label{Schematische Darstellung von einem beispielhaften Regressionsnetz}
    \quelle\url{https://miro.medium.com/1*LaEgAU-vdsR_pClMcgbikQ.jpeg}
\end{figure}


In vereinfachte darstellung der Schichten eines Neurales Netzes ist in Abbildung \ref{Schematische Darstellung von einem beispielhaften Regressionsnetz}. Eine Schicht an künstlichen Neuronen ist so definiert, 
dass man die Neuronen der Schicht immer mit der gleichen Anzahl an Kantenübergägen erreicht. Jedes Neuron der ersten verborgenen Schicht kann zum Beispiel mit nur genau einem Kantenübergang erreicht werden.
Für eine vollständig verbundene Schicht muss jedes künstliche Neuron mit jedem künstlichen Neuron der vorherigen und drarauffolgenden Schicht verbunden sein. Dabei gibt es drei Arten von Schichten, wie in der Abbildung \ref{Regression auf Merkmalsvektor} zu sehen. Zuerst gib es die Eingabeschicht.
Diese Schicht repräsentiert die Eingaben und ist in Abbilding \ref{Schematische Darstellung von einem beispielhaften Regressionsnetz} rot markiert. Jeder Knoten der Eingabeschicht ist ein Datenpunkt, oder in diesem Fall ein Merkmal. 
Danach gibt es die versteckten Schichten, in Blau markiert. Ihre Aufgabe ist es die Eingabe zu transformieren. 
Sie erlauben es dem Neuralen Netz eine Eingabe über mehrere Schichten zu verarbeiten. Durch kann das Neurale Netz nicht lineare zusammenhänge darstellen. Zuletzt gibt es die Ausgabeschicht. Von dieser Schicht wird die Z-Invarianz abgelesen.

\subsection{Merkmalsextraktion aus Punktwolke}
\label{Merkmalsextraktion aus Punktwolke}

Wenn eine Eingabe ohne Umwege auf die Ausgabe abgebildet werden soll, muss ein tiefes Netz eingesetzt werden. Um ein Bauteil automatisch auszurichten, ist es notwendig, dass das Neurale Netz die Eingabedaten selber verarbeitet.
Der Unterschied zwischen einem tiefen und nicht tiefen Neuralen Netz ist es, dass ein tiefes Neurales Netz nicht nur die Abbildung von Merkmalen auf Ausgaben erlernt, sondern auch die Extraktion von Merkmalen aus den Eingabedaten. 
Die Extractrion der Merkmale wird in diesem Abschnitt erklährt.
In der Merkmalsextraktion wird eine Punktwolke auf eine Anzahl an Merkmalen reduziert. Die Merkmale, auf welche die Punkwolke reduziert wird, werden in einem flachen Vektor $feat = (v_{0},v_{1},v_{2},\ldots,v_{n})$ mit $v_{i} \in \mathbb{R}$ der Größe n repräsentiert.
Dieser Vektor wird Merkmalsvektor genannt und ist eine abstrakte Darstellung der Punktwolken. Der Merkmalsvektor ist als Zwischenergebniss wichtig, um die Eingabedaten in für ein Regressionsnetz besser verarbeitbare Form zu bringen. 
Das Regressionsnetz ist der zweite Teil der Struktur des Neurales Netzes und wird in dem folgenden Abschnitt \ref{Regression auf Merkmalsvektor} beschrieben.

\smallskip

Die Struktur der Merkmalsextraktion besiert auf der Arbeit:  \cite{test}

\smallskip

Die generelle Struktur der Merkmalsextraktion ist eine wiederholte Folge an Reduzierungen auf den konkreten Daten der Punktwolke, gefolg von einem kleinem voll verbundenen Neuralen Netz, was mit den konkreten Daten dieser Schicht Merkmale über mehrere Schichten hinweg konstruiert.
In dem ersten Schritt in jeder Schicht werden Punkte aus der Punktwolke ausgewählt. Dafür wird ein Algorithmus eingestzt, der eine Teilmenge der Ausgangspunkte auswählt, um eine Möglichst gute Abdeckung über die Augangspunkte zu erreicht (Farthest point sampling).

\begin{figure}[t]
    \centering
    \includegraphics[scale=0.28]{bilder/Pointnet++.jpg}
    \caption[Schematische Darstellung von Merkmalsextraktionschichten]{Schematische Darstellung von Merkmalsextraktionschichten}
            \label{Schematische Darstellung von Merkmalsextraktionschichten} 
    \quelle\url{https://stanford.edu/~rqi/pointnet2/images/pnpp.jpg}
\end{figure}

Dafür geht der Algorithmus schrittweise vor. In jedem Schritt wählt er den Punkt der Punktwolke aus, der am weistesten von allen bisher ausgewählten Punkten entfernt ist, solange bis die gewünschte Menge an Punkten erreicht ist.
Der Algorithmus sucht nicht nach der optimalen Abdeckung, welche die Distanz von jedem nicht ausgewählten Punkt zu dem nächsten ausgewählten Punkt minimierent, sondern ist nur eine Abschätzung, dieser optimalen Lösung.
Die gefundenen Punkte bilden die ausgewählten Punkte für die aktuelle Schicht. Danach werden für jeden ausgewählten Punkt jeder Punkt innerhalb eines Radius um sich selber verbunden. Diese Punkte sind die Nachbarn des ausgewählten Punktes.
In der Abbildung \ref{Schematische Darstellung von Merkmalsextraktionschichten} wir die Auswahl von Punkten und die Reduktion der ausgewählten Punkte in dem "Sampling and Grouping" Schritt gezeigt. Es ist zu sehen, 
dass um bestimmte Punkte nach diesem Schritt ein Radius gezeichnet ist, der die Nachbarn der ausgewählten Punkte darstellt.

\begin{center}
\begin{math} Auswahl \Subset Punktwolke\end{math}
\end{center}

\begin{figure}
    \centering
    \begin{math} \forall \vec{a} \in Auswahl , \forall \vec{p} \in Punktwolke: \left|  \left| \vec{a} - \vec{p}  \right| \right|_{2} \leq r \implies (p,a) \in Kanten \end{math}
\end{figure}

Die ausgewählten Punkte werden als künstliche Neuronen genutzt. Dabei sind die Merkmale, die in den Punkten der Nachbarn gespeichert sind, die Eingabe für das künstliche Neuron. Die Gewichte für jede Eingabe werden erlernt, dafür wird ein Multi Layer Perceptor (MLP) genutzt.
Der MLP lernt, wie die Gewichte für die oben bestimmten Kanten seinen sollen. Dafür Nutz der MLP die relative Position von ausgewählten Punkten zu ihren Nachbarn. Ein MLP funktioniert ähnlich zu dem Regressionsnetz, was in Abschnitt \ref{Regression auf Merkmalsvektor} erklährt wird.
Ein ausgewählter Punkt verändert seine Merkmale anhand der Summe von Mermalen der Punkte in seiner Nachbarschaft, die nach ihrer relativen Position gewichtet werden. Die Anzahl an Punkten verringert sich, aber die Anzahl der Merkmale pro Punkt erhöht sich mit jeder Schicht.

Über mehrere Schichten hinweg, werden Merkmale für die Punkte gelernt, um dann die Punkte zu redutieren und die Merkmale an die ausgewählten Punkte weiterzugeben. Das generiert eine Menge an Merkmalen für jeden Punk, 
aber für die Weiterverarbeitung wird ein Merkmalsvektor benötigt, der das gesammte Bauteil beschreibt. Im letzten Schritt wird das größete Merkmal in jeder Dimension der Merkmale aller Punkte genommen. So wird ein flacher Merkmalsvektor erstellt.
Mit dem größten Merkmal ist das numerisch Größte gemeint, nicht umbedingt das Wichtigste. 

\subsection{Regression auf Merkmalsvektor}
\label{Regression auf Merkmalsvektor}

Nach der Bestimmung von abstrakten Merkmalen in einem Merkmalsvektor im vorherigen Abschnitt \ref{Merkmalsextraktion aus Punktwolke} muss aus diesem Vektor ein konkretes Ergebnis abgeleitet werden. Dafür wird eine Abbildung von dem Vektor auf die Ergebnisse erlernt. Dieses Vorgehen wird Regression genannt. 
In diesem Fall wird auf die Z-Invarinaz regressiert.

\smallskip

Eine Schicht funktioniert so, dass jedes Neuron die gewichtetete Summe von allen vorhergeneden Neuronen erhält und diese Summe an eine Aktivierungsfunktion übergibt. Dabei werden Methoden wie Normalisierung und zufälliges zurrücksetzten von Eingaben verwendet, um
die Transformation verläslicher zu machen.

Die Normalisierung stellt sicher, dass es keine zu großen Änderungen innherlab einer Schicht vorkommen, da alle Werte relativ zu dem Durchschnitt normaliesiert werden. Bei einer großen Änderung innerhalb einer Schicht, zwingt es darauffolgende Schichten ebenfalls Ändrungen vorzunehmen. 
Die konstante Anpassung zu Änderungen in den vorherigen Schichten macht das Lernen instabiel.

Das Zufällige zurrücksetzen von Eingaben verhinder eine zu große Abhängigkeit von nur einem Neuron oder einem Pfad. Dafür werden mit einer vorbestimmten Warscheinlichkeit gewichtetete Summen auf null gesetzt, sodass sie keinen Einfluss auf die nächste Schicht haben.


\section{Konstruktion der Ausrichtung}
\label{Konstruktion der Ausrichtung} % wie wird aus z Invarianz und Z drehung eine Rotation (und warum)

Die, in Abschnitt \ref{Regression auf Merkmalsvektor} berechnete, Z-Invarianz ist nicht genug, um eine Drehung darszustellen, um ein Bauteil drehen zu können ist eine Rotationsmatrix notwendig. 
Rotationsmatrizen sind, für den 3d Raum, 3 x 3 Matrizen, die durch Matrixmultiplikation eine Punktwolke drehen können.

\begin{figure}[H]
    \centering
    \[ 
    \begin{bmatrix}  
    x_1 & y_1 & z_1 \\  
    x_2 & y_2 & z_2 \\  
    \vdots & \vdots & \vdots \\  
    x_n & y_n & z_n \\  
    \end{bmatrix}
    \begin{bmatrix}  
    R_{11} & R_{12} & R_{13} \\  
    R_{21} & R_{22} & R_{23} \\  
    R_{31} & R_{32} & R_{33}
    \end{bmatrix}  
    =
    \begin{bmatrix}  
    x_1' & y_1' & z_1' \\  
    x_2' & y_2' & z_2' \\  
    \vdots & \vdots & \vdots \\  
    x_n' & y_n' & z_n' \\  
    \end{bmatrix}  
    \]
\end{figure}

Dabei ist die Norm von jeder Zeile gleich Eins und jede Zeile ist, wenn man sie als Vektor betrachtet orthogonal zu den anderen Zeilen. Dasselbe gilt für Spalten.

\begin{figure}[H]
    \centering
    $ \forall i \in \{1 , 2, 3 \}: \left|  \left| \begin{bmatrix} R_{1i} \\ R_{2i} \\ R_{3i} \end{bmatrix} \right| \right|_{2} = 1 $ \hspace{1cm}und\hspace{1cm}
    $ \forall i \in \{1 , 2, 3 \}: \left|  \left| \begin{bmatrix} R_{i1} \\ R_{i2} \\ R_{i3} \end{bmatrix} \right| \right|_{2} = 1$
\end{figure}

Um aus dem 3d Vektor, der die Z-invarianz darstellt, eine Rotationsmatrix zu erstellen, kann die Z-Invarianz selber alz Zeile der Rotationmatrix genommen werden. Die restlichen Zeilen müssen dann aus der ersten Zeile abgeleitet werden.
Die Zeilen einer Rotationmatrix müssen Orthogonal zueinander sein, daher muss für die anderen Zeilen Vektoren gefunden werden die Orthogonal zu der Z-Invarianz sind. Wenn zwei Vektoren gegeben sind, kann man aus ihenen eine Ebene bilden 
eine Normale dieser Ebene ist Orthogonal zu beiden Vektoren. Dadurch lassen sich orthogonale Vektoren zu der Z-Invarinaz finden.



    \tdplotsetmaincoords{70}{110}
    \begin{figure}[t]
        \begin{tikzpicture}[scale=4,tdplot_main_coords]
            \draw[thick,->] (0,0,0) -- (1,0,0) node[anchor=north east]{$x$};
            \def\x{.5}
            \filldraw[
                draw=red,%
                fill=red!20,%
            ]          (0,0,0)
                    -- (\x,{sqrt(3)*\x},0)
                    -- (\x,{sqrt(3)*\x},1)
                    -- (0,0,1)
                    -- cycle;
            \draw[thick,->] (0,0,0) -- (0,1,0) node[anchor=north west]{$y$};
            \draw[blue,very thick, ->] (0,0,0) -- ({1.2*\x},{sqrt(3)*1.2*\x},0) node[below] {$Z-Invarianz$};
            \draw[blue,very thick,->] (0.3,0.51,0.5) -- ({sqrt(3)*0.6*\x + 0.3},{-0.6*\x + 0.51},0.5) ;
            \draw[blue,very thick,->] (0,0,0) -- ({sqrt(3)*0.6*\x},{-0.6*\x},0) node[anchor=north]{$n_1$};
            \draw[thick,dotted] ({sqrt(3)*0.6*\x + 0.3},{-0.6*\x + 0.51},0.5) -- ({sqrt(3)*0.6*\x},{-0.6*\x},0);
            \draw[thick,dotted] (0,0,0) -- (0.3,0.51,0.5);
            \draw[blue,very thick,->] (0,0,0) -- (0,0,1) node[anchor=south]{$z / Hilfsvektor$};

        \end{tikzpicture}
        \caption[Berechung von $n_1$]{Berechung von $n_1$}
            \label{Berechung von $n_1$}
    \end{figure}

Da die Z-Invarianz aber nur ein Vektor ist, muss für die Bestimmung anderer Zeilen im ersten Schritt ein Hilfvektor genommen werden. Dafür bietet sich die z-Achse an, da die Rotationen relativ zu der z-Achse bewertet werden.
Zwischen der Z-Invarinaz und der z-Achse wird eine Ebene aufgespannt. Die Normale der Ebene ist eine Zeile in der Rotationmatrix.

\smallskip

Als Hilfvektor wird immer die normierte z-Achse genommen, daher besteht die Möglichkeit, dass die Z-Invarinaz sehr ähnlich zu der z-Achse ist. Das macht die berechnug der Normalen mathematisch instabiel, da durch die extreme Änlichkeit jetzt der Nullvektor als Ergebnis kommen könnte. 
Um dieses Problem zu beheben, wird überprüft, ob der Hilsvetor zu ähnlich zu der Z-Invarianz ist. Wenn sie zu ähnlich sind, wird der Hilfvektor durch einen anderen Vektor ersetzt. Einen anderen Hilfsvektor zu nehmen beeinfluss die Rotation, aber diese Ausname ist sehr selten, verschlechtert sie das durchschnittliche Ergebniss nicht sehr.
Als Alternative zu der z-Achse bieten sich die x und y Achsen an, da sie beide 90 Grad von der z-Achse entfernt sind und dadurch, die Struktur der Rotation erhalten bleibt.


    
    \tdplotsetmaincoords{70}{110}
    \begin{figure}[t]
        \centering
        \begin{tikzpicture}[scale=4,tdplot_main_coords]
            \def\x{.5}
            \filldraw[
                draw=red,%
                fill=red!20,%
            ]          (0,0,0)
                    -- (\x,{sqrt(3)*\x},0)
                    -- ({\x + sqrt(3)*\x},{sqrt(3)*\x + -1*\x},0)
                    -- ({sqrt(3)*\x},{-1*\x},0)
                    -- cycle;
            \draw[thick,->] (0,0,0) -- (1.5,0,0) node[anchor=north east]{$x$};
            \draw[thick,->] (0,0,0) -- (0,1,0) node[anchor=north west]{$y$};
            \draw[thick,->] (0,0,0) -- (0,0,1) node[anchor=south]{$z$};
            \draw[blue,very thick, ->] (0,0,0) -- ({1.2*\x},{sqrt(3)*1.2*\x},0) node[below] {$Z-Invarianz$};
            \draw[blue,very thick,->] (0,0,0) -- ({sqrt(3)*1.2*\x},{-1.2*\x},0) node[anchor=north]{$n_1$};
            \draw[blue,very thick,->] (0,0,0) -- (0,0,0.5) node[anchor=east]{$n_2$};
            \draw[blue,very thick,->] (0.8,0.2,0) -- (0.8,0.2,0.5) ;
            \draw[thick,dotted] (0,0,0.5) -- (0.8,0.2,0.5);
            \draw[thick,dotted] (0,0,0) -- (0.8,0.2,0);
        \end{tikzpicture}
        \caption[Berechung von $n_2$]{Berechung von $n_2$}
            \label{Berechung von $n_2$}
    \end{figure}

Durch das Berechnen einer anderen Zeile außer der Z-Invarianz ist ein Hilfsvektor nicht mehr vonnöten. Die berechnete Normale bildet die letzte Zeile, die für eine Rotationmatrix vonnöten ist. Sie ist orthogonal zu den anderen Zeilen, da sei eine Normale in der Ebene der anderen Zeilen ist.
Normale-1 ist ebenfalls orthogonal zu der Normale-2, da es in der Ebene ist, desssen normale Normale-2 ist. Sie ist ebenfalls otrhogonal zu der Z-invarianz, da Normale-1 aus Ebene, die die Z-Invarianz beinhaltet gebildet wurde. 
Orthogonalität ist symetrisch, daher ist auch die Z-Invarianz orthogonal zu Normale-1 und Normale-2. Das erfüllt die oben genannte Bedingenung, dass alle Zeilen orthogonal zueinander seien müssen.



    \begin{figure}[t]
        \centering
        \subfigure[]
            {\includegraphics[scale=0.25]{bilder/Rotation_V1}\label{Rotation_V1}
        }
        \subfigure[]
            {\includegraphics[scale=0.25]{bilder/Rotation_V2}\label{Rotation_V2}
        }
        \caption[Vertauschen von $n_1$ und $n_2$ in der Rotationsmatrix]{Vertauschen von $n_1$ und $n_2$ in der Rotationsmatrix}
            \label{Vertauschen von $n_1$ und $n_2$ in der Rotationsmatrix}
    \end{figure}

Die Z-Invarianz wird als unterste Zeile genommen, die beiden berechneten Normalen können eine beliebige andere Zeile nehmen. Die Wahl der Zeilen verändert nur die Rotation des Bauteils um die z-Achse herum, wie man in der Abbildung \ref{Vertauschen von $n_1$ und $n_2$ in der Rotationsmatrix} sehen kann. 
Solange also die Zeilen immer gleich angeordnet werden, wird die Gradabweichung von der z-Achse gleich bleiben.

\smallskip

Nachdem die Rotationmatrix erstellt wurde muss zuletzt überprüft werden, on sie eine gültige Rotation ist. Im 5d-Druck kann sich die Platform nicht völlig frei bewegen, sie kann eine maximale Gradabweichung vom Ursprung von 90 Grad unterstüzen, also insgesammt jeweils 180 Grad entlang zwei Achsen.
Daher wird überprüft, ob die Rotationsmatrix das Bauteil über diese Grenzen hinaus drehen würde. Falls dem so ist, muss die Rotationsmatrix gespiegelt werden.



    \begin{figure} [t]
        \centering
        \[
        \begin{bmatrix}  
        1 & 0 & 0 \\  
        0 & 1 & 0 \\  
        0 & 0 & -1
        \end{bmatrix}
        \]
    \end{figure}

    \begin{figure}[t]
        \centering
            \subfigure[ohne Spiegelmatrix]
                {
                \begin{tikzpicture}[scale=2.5,tdplot_main_coords]
                \def\x{.5}
                \draw[thick,->] (0,0,0) -- (1.5,0,0) node[anchor=north east]{$x$};
                \draw[thick,->] (0,0,0) -- (0,1,0) node[anchor=north west]{$y$};
                \draw[thick,->] (0,0,0) -- (0,0,1) node[anchor=south]{$z$};

                \draw[blue,thick,->] (0,0,0) -- (0.6841, -0.6408, -0.3483) node[anchor=south]{};
                \draw[blue,thick,->] (0,0,0) -- (0.0000, -0.4776,  0.8786) node[anchor=south]{};
                \draw[blue,thick,->] (0,0,0) -- (0.7294,  0.6011,  0.3267) node[anchor=south]{};
            \end{tikzpicture}
            }
            \subfigure[mit Spiegelmatrix]
                {
                \begin{tikzpicture}[scale=2.5,tdplot_main_coords]
                \def\x{.5}
                \draw[thick,->] (0,0,0) -- (1.5,0,0) node[anchor=north east]{$x$};
                \draw[thick,->] (0,0,0) -- (0,1,0) node[anchor=north west]{$y$};
                \draw[thick,->] (0,0,0) -- (0,0,1) node[anchor=south]{$z$};

                \draw[blue,thick,->] (0,0,0) -- (0.6841, -0.6408, -0.3483) node[anchor=south]{};
                \draw[blue,thick,->] (0,0,0) -- (0.0000, 0.4776,  -0.8786) node[anchor=south]{};
                \draw[blue,thick,->] (0,0,0) -- (0.7294,  0.6011,  0.3267) node[anchor=south]{};
            \end{tikzpicture}
            }
            \caption[Veränderung der Rotationmatrix durch die Spiegelmatrix]{Veränderung der Rotationmatrix durch die Spiegelmatrix}
            \label{Veränderung der Rotationmatrix durch die Spiegelmatrix}
    \end{figure}

Durch die Matrixmultiplikation mit der Spiegelmatrix, aus \ref{Veränderung der Rotationmatrix durch die Spiegelmatrix}, wird eine Rotationsmatrix in der xy-Ebene gespiegelt. Das hat den Effekt, dass sich das Bauteil um 180 Grad dreht. Da eine Spiegelung nötig war, galt vor der Spiegelung, dass der Betrag der Rotation mehr als 90 Grad ist. 
Die Rotation ist also zwischen 90 und 180 Grad. Sie kann nicht höher sein, da eine sinnvolle Rotation nur von -180 bis 180 Grad gehen kann und so einen ganzen Kreis bildet. Wenn man die Rotation spiegelt, wird sie um 180 Grad gedreht, es gilt: 

\begin{figure}[t]
    \centering
    $rot \in \mathbb{Q} \bigwedge \vert rot \vert > 90 \bigwedge \vert rot \vert < 180 \implies \vert rot \vert - 180 < 90$
\end{figure}

Wenn sie vor der Spiegelung nicht im 180 Grad Halbkreis der erlaubten Drehungen war, dann ist sie es danach.



\cleardoublepage
%========================================================================================
% TU Dortmund, Informatik Lehrstuhl VII
%========================================================================================

\chapter{Evaluation}
\label{Evaluation}

Dei Ausrichtung eines Bauteils ist in den meisten Fällen nicht perfekt. In diesem Kapitel wird behandelt wie gut das in Kapitel \ref{Punktwolkenverarbeitung zur Bauteilausrichtung} erstellte Neurales Netz ein Bauteil ausrichten kann.
Dafür wird nicht nur die Ausrichung selber überprüft. Es wird auch die größe der Rotationen untersucht, da das Drehen eines Bauteil Zeit kostet und die minimierung diser Drehung wie in \ref{Fertigungs-Geschwindigkeit} erwähnt eine Priorität ist.
Außerdem wird die Fähigkeit des neuralen Netzes Fehler in den Daten zu wiederstehen in Abschnitt \ref{Verlässlichkeit der Ausrichtung} überprüft und die Fähigkeit auf komplexen Bauteilen zu arbeiten in Abschnitt \ref{Generalisierbarkeit} getestet.



\begin{figure}[h]
    \centering
    \includegraphics[scale=0.2]{bilder/Daten.png}
    \caption[Darstellung aller Typen von Bauteilen im Datensatz]{Darstellung aller Typen von Bauteilen im Datensatz}
            \label{Darstellung aller Typen von Bauteilen im Datensatz}
    \quelle\url{https://media.springernature.com/full/springer-static/image/art%3A10.1007%2Fs40964-025-00960-6/MediaObjects/40964_2025_960_Fig1_HTML.png}
\end{figure}

Alle Daten, die für die Auswertung benutzt werden, sind in dem Artickel: \cite{Datensatz} erstellt worden und dann wie in Kapitel \ref{Flächensuche in  Bauteilen} beschrieben verändert worden.
Die Abbilding \ref{Darstellung aller Typen von Bauteilen im Datensatz} stellt alle Typen von Bauteilen dar, die in dem Datensatz enthalten sind. Es gebt insgesammt 4 verschiedene Typen, die alle einfache geometrische Objekt sind.

Die Typen sind von A bis D durchnummeriert. Die Unterschiedlihen Type sind: "Platten mit Löchern", "Platonischer Körper mit Inschriften", "Körper mit zwei Löchern" und "Pyramieden".

Für alle Tests, in dem Abschnitt \ref{Effekt von Rauschen auf den Daten} und \ref{Effekt von schlechter Flächensuche}, werden 1000 zufällige Flächen aus dem Datensatz genommen. Diese Flächen werden wie in Kapitel \ref{Punktwolkenverarbeitung zur Bauteilausrichtung} beschrieben gefunden. 
Tests haben ergeben, dass das Verwenden von einem Datensatz mit mehr als 1000 Flächen keinen Einfluss auf das Ergebniss hat. 

\section{Ausrichtung der Bauteile}
\label{Ausrichtung der Bauteile}

Die Ausrichtung der Bauteile ist das Ziel der Arbeit. Dementsprechend ist es die wichtigste Metrik für den Erfolgt des neuronalen Netzes. 
Die Minimierung der Bewegung ist ebenfalls wichtig, da jede Drehung eines Bauteils Zeit kostet. Als zweites Ziel des Netzes wird dementsprechend versucht, die Größe der Drehung des Bauteils zu minimieren.

\begin{table}[h]
\centering
\begin{tabular}{lrr}
\toprule
Bauteil & Gradabweichung vom Optimum & Bewegung des Bauteils\\
\midrule
Platten mit Löchern & 10.12 & 23.56 \\ \addlinespace
Platonischer Körper mit Inschriften & 9.59 & 26.50 \\ \addlinespace
Körper mit zwei Löchern & 9.40 & 27.11\\
Pyramieden & 9.22 & 24.68\\
\bottomrule
\end{tabular}
\caption{Testergebnisse auf einfachen Bauteilen}
\label{table:Testergebnisse auf einfachen Bauteilen}
\end{table}

Die Tabelle \ref{table:Testergebnisse auf einfachen Bauteilen} zeigt die Ergebnisse der Tests auf den einfachen Bauteilen des Datensatzes. Dabei ist jeder Typ von Bauteil einzeln aufgeführt.
Es lässt sich erkennen, dass alle Bauteile außer die Platten mit Löchern eine Genauigkeit von unter 10 Grad Abweichung vom Optimum erreichen. Die Platten mit Löchern erreichen eine Genauigkeit von 10.12 Grad Abweichung vom Optimum, was nicht viel höher ist.
Insgesammt lässt sich kein größer Unterschied zwischen den verschiedenen Typen von Bauteilen erkennen. Alle Typen erreichen eine ähnliche Genauigkeit und benötigen eine ähnliche Bewegung des Bauteils.
Die Pyramieden erreichen die beste Genauigkeit mit einer Abweichung von 9.22 Grad vom Optimum und einer durchschnittlichen Bewegung von 24.68 Grad. Die Pyramieden sind die einfachsten Bauteile im Datensatz, was die gute Genauigkeit erklären könnte.
Andererseits sind die Platten mit Löchern die am schlechtesten abschneidenden Bauteile, was die Gradabweichung vom Optimum angeht. Das könnte daran liegen, dass die Flächensuche Probleme mit den Löchern in den Platten hat.

\section{Verlässlichkeit der Ausrichtung}
\label{Verlässlichkeit der Ausrichtung}

Ein wichtiger Unterschied zwishen dem Trainigsdatensatz und der Praxis ist, dass man in der Praxis nicht davon ausgehen kann, dass die Daten einwandfrei sind. Um ein Praxisrelevantes Verfahren zu entwickeln, muss das Verfahren in der Lage sein, Fehler in den Daten zu widerstehen.
In dem bisherigen Test ist immer davon ausgegangen worden, dass die Punktwolken eine perfekte Darstellung eines Bauteils ist. In der Praxis, muss aber davon ausgegangen werden, dass die Punktwolken Fehler aufweisen. Um diese Fehler zu simulieren, 
wird in den folgenden zwei Abschnitten der Einfuss von fehlenden Punkten und der Einfluss von falschen Punkten untersucht. 

\subsection{Effekt von Rauschen auf den Daten}
\label{Effekt von Rauschen auf den Daten}

Eine Fläche eines Bauteils, die Punkte beinhaltet, die nicht zu der Fläche gehören, kann das Ergebnis des neuronalen Netzes stark beeinflussen. Dieser Fehler könnte durch schlechtes Clustering entsehen, wenn zwei Flächen nicht richtig getrennt werden.
Punkte die nicht zu der Fläche gehören, aber dennoch in der Punktwolke der Fläche enthalten sind, werden als Rauschen (engl. Noise) bezeichnet. Um den Einfluss von Rauschen auf das Netz zu bestimmen, werden Punkte zu sonst korrekten Flächen hinzugefügt.

\begin{figure}[h]
    \centering
    \includegraphics[scale=0.5]{bilder/Rauschen2.png}
    \caption[Test mit Noise auf dem Neuralen Netz]{Test mit Noise auf dem Neuralen Netz}
            \label{Test mit Noise auf dem Neuralen Netz}
\end{figure}

Die Abbildung \ref{Test mit Noise auf dem Neuralen Netz} zeigt die Ergebnisse des Tests. Um Noise zu simulieren, wurden für jede Fläche um das Zentrum der Fläche zufällig Punkte verteilt. 
Dabei wurden die Punkte normalverteilt um das Zentrum der Fläche, sodass die meisten Punkte nahe der Fläche sind. Die zufällige Verteilung bricht die köhärent Struktur der Fläche auf. In der Abbildung \ref{Test mit Noise auf dem Neuralen Netz} 
wurde der Unterschied des Normalvektors der Fläche zu der z-Achse in Orange aufgezeichnet,
die Standartabweichung des Normalvectors zu der z-Achse in Grün und die benötigte Größe der Drehung ist Blau. Die x-Achse der Abbilding zeigt das Rauschen. Dabei ist die Anzahl der Rauschpunkte gleich des Wertes auf der x-Achse mal 20. 
Die y-Achse zeigt die Ergebnisse bei den jeweiligen Rauschleveln. Abbilding \ref{Test mit Noise auf dem Neuralen Netz} zeigt, 
dass das Rauschen keinen großen Einfluss auf den Unterschied des Normalvektors der Fläche zu der z-Achse oder die benötigte Größe der Drehung hat. Auf die Standartabweichung des Normalvektors hat das Rauschen allerdings einen extremen Einfluss.
Sobald Rauschen hinzugefügt wird, steigt die Standartabweichung extrem an und bleibt dann bei dem Fehlermaß, zu dem es steigt unabhängig von weiterem Rauschen, dass hinzugefügt wird. Daraus lässt sich schließen, dass das Netz keine gute Fähigkeit hat, 
Rauschen zu widerstehen. Die absolute Ausrichtung in Orange bleibt zwar stabiel, aber da die Abwichung von diesem Durchschnitt extrem ansteigt, ist das Ergebnis unzuverlässig.

\smallskip

Der extreme Anstieg, der Standartabweichung in Abbilding \ref{Test mit Noise auf dem Neuralen Netz} könnte dadurch erklährt werden, dass das Netz nur auf fehlerfreien Daten trainiert wurde. Die Rauschpunkte eleminieren die glatte Struktur der Fläche, 
aber die Merkmalsextraktion wird dennoch versuchen diese Punkte mit einzubeziehen und dadurch auf falsche Merkmale kommen.  Das erklährt ebenfalls, warum die Standartabweichung nach dem ersten Anstieg konstant bleibt. Sobald die Merkmale falsch sind,
ändert weiteres Rauschen nichts mehr an den falschen Merkmalen. Die Ergebnisse sind dennoch besser als Zufällig, was darauf hindeutet, dass das Netz trotz des Rauschens noch die Originale Fläche erkennt, 
aber die Rauschpunkte eine große Unsicherheit in die Ergebnisse bringen.

\subsection{Effekt von schlechter Flächensuche}
\label{Effekt von schlechter Flächensuche}

Eine Fläche, die in einem Bauteil gefunden wurde, muss nicht immer perfekt die Fläche des Bauteils repräsentieren. Häufig werden Punkte, die zu der Fläche gehören sollten, nicht gefunden, was die Punktwolke keiner macht als sie es mit einer perfekten Suche wäre.
Der Einfluss von fehlenden Punkten in der Punktwolke einer Fläche wird getestet, indem über mehrere Iterationen Punkte aus den Punktwolken entfernt werden. Die Anzahl der Punkte basiert auf einem Prozentsatz, der sich mit jeder Iteration erhöht. 
Durch das Verwenden von Prozentsätzen wird sichergestellt, dass die Ergebnisse unabhängig von der ursprünglichen Größe der Punktwolke sind.

\begin{figure}[h]
    \centering
    \includegraphics[scale=0.5]{bilder/Entfernen.png}
    \caption[Test von schlechter Flächensuche auf dem Neuralen Netz]{Test von schlechter Flächensuche auf dem Neuralen Netz}
            \label{Test von schlechter Flächensuche auf dem Neuralen Netz}
\end{figure}

Die Abbildung \ref{Test von schlechter Flächensuche auf dem Neuralen Netz} zeigt den Verlusst an Genauigkeit des Netzes, wenn Punkte aus der Punktwolke entfernt werden. Alle Ergenisse sind in Grad angegeben. Auf der x-Achse ist der Prozentsatz der entfernten Punkte dargestellt, 
auf der y-Achse ist das Ergebnis der Tests dargestellt. Die Abweichung von dem Normalvektor der Fläche ist nahezu unverändert, wenn Punkte entfernt werden. Die Standartabweichung des Normalvektors bleibt für die ersten 80 \%, ebenfalls
fast unverändert, und steigt nur leicht an. Erst nachdem 80 \% der Punkte entfernt wurden, steigt die Standartabweichung stark an. Bei 90 \% entfernten Punkten ist die Standartabweichung fast bei 45 Grad angekommen, was bedeutet, das bei einem so hohen Verlusst das Ergebnis kaum besser als Zufall ist.
Aber bis zu diesem Punkt zeigt das Netz eine gute Fähigkeit, fehlende Punkte in der Punktwolke zu widerstehen. Ein Verlusst von über 80 \% der Punkte ist extrem unwahrscheinlich in der Praxis. Das netz zeigt also eine gute Wiederstandfähigkeit gegen schlechte Flächensuche.

\smallskip

Auch bei der Differenz zu der z-Achse oder der Bewegung des Bauteils ist kein großer Einfluss von fehlenden Punkten zu erkennen, bis 80 \% der Punkte entfernt wurden. 
Danach hat auch hier der Verlust an Punkten einen großen Einfluss auf die Genauigkeit des Netzes, allerdings ist der Anstieg deutlich geringer im Vergleich zu der Standartabweichung.
Der Punkt, an dem die Genauigkeit stark abfällt, ist bei der Standartabweichung und der Differenz zu der z-Achse gleich. Das deutet darauf hin, dass der Abfahl der Genauigkeit denselben Grund hat.

\smallskip

Der Zerfall der Genauigkeit in Abbildung \ref{Test von schlechter Flächensuche auf dem Neuralen Netz} könnte dadurch erklährt werden, dass die Merkmalsextraktion darauf basiert, dass eine Menge an Punkten Merkmale besitzt, die in jedem Schritt auf eine kleinere Menge an Punkten reduziert werden.
Wenn zu viele Punkte fehlen, dann wird das Netz weniger Punkte für die Reduktionen haben, was den langsamen Zerfall der Standartabweichung bis zu der 80 \% Grenze erklären könnte. Warum es danach so stark abfällt, ist vermutlich damit zu erklähren, 
dass die Dichte der Punkte so stark abgenommen hat, dass die Reduktion häufig keine Punkte in der loken Umgebung hat, die sie für eine Reduzierung verwenden kann. Wenn die Merkmale von einem oder sehr wenigen Punkten aus reduziert werden, 
denn zerfällt die Idee, locale Merkmale anhand Ihrer relativen Position zu andrern Punkten zu bewerten. Das neurale Netz verliert die Fähigkeit, locale Merkmale zu erkennen.

\section{Generalisierbarkeit}
\label{Generalisierbarkeit}

Alle bisherigen Auswertungen wurden auf den gleichen Typen von Daten durchgeführt, auf denen das Netz auch trainiert wurde. Um die Verlässlichkeit des Netzes zu überprüfen wird in diesem Abschnitt getestet, ob das Netz auch auf komplexeren Bauteilen arbeiten kann.
Für diesen Zweck wurden mehrere komplexe Bauteile getestet, komplex bedeutet in diesem Fall, dass die Bauteile nicht aus einfachen geometrischen Formen bestehen, sondern gebogene Flächen und komplexe Strukturen beinhalten.

\begin{table}[h]
\centering
\begin{tabular}{lrr}
\toprule
Bauteil & Gradabweichung vom Optimum & Bewegung des Bauteils\\
\midrule
Stanford Hase & 10.76 & 17.29 \\ \addlinespace
Stanford Drache & 11.61 & 25.02 \\ \addlinespace
Utah Teapot & 10.84 & 26.72\\
\bottomrule
\end{tabular}
\caption{Testergebnisse auf komplexen Bauteilen}
\label{table:Testergebnisse auf komplexen Bauteilen}
\end{table}

Alle Angaben in der Tabelle \ref{table:Testergebnisse auf komplexen Bauteilen} sind Durchschnittswerte über alle Flächen eines Bauteils. Die Ergebnisse sind immer in Grad und für jedes Bauteil einzeln angegeben. 
Es lässt sich erkennen, dass die Abweichung vom Optimum bei komplexen Bauteilen höher ist, als bei den einfachen Bauteilen, die in Abschnitt \ref{Ausrichtung der Bauteile} getestet wurden. Die Abweichung vom Optimum liegt bei allen komplexen Bauteilen bei etwa 11 Grad.
Der geringe Unterschied zwischen den Ergebnissen der komplexen Bauteile lässt darauf schiezen, dass die Auswahl der komplexen Bauteile eine gute Repräsentation für komplexe Bauteile im Allgemeinen ist. 
Allerding ist die Bewegung des Bauteils bei dem Stanford Hasen deutlich geringer als bei den anderen beiden Bauteilen. Der Unterschied könnte daran liegen, dass der Stanford Hase nur eine Gundfläche hat. Grundfächen werden immer für hohe Bewegung sorgen, 
da die optimale Ausrichtung meistens im 90 Grad Winkel zu der Grundflächge liegt. Der Stanford Drache und die Utah Teapot haben beide mehrere Grundflächen, was die Bewegung des Bauteils erhöhen könnte.



\cleardoublepage
%========================================================================================
% TU Dortmund, Informatik Lehrstuhl VII
%========================================================================================

\chapter{Zusammenfassung und Ausblick}
\label{Zusammenfassung und Ausblick}

\section{Zusammenfassung}
\label{Zusammenfassung}

In dieser Arbeit wurde zunächst die Notwendigkeit der Ausrichtung von Bauteilen in der additiven Fertigung erläutert. Dabei wurde ganz besonders auf die zusätzlichen Herausforderungen bei dem 5D Druck eingegangen, in dem die Ausrichtung eine größere Rolle spielt, 
da die Ausrichtung eines Bauteils mehrmals während des Druckens geändert werden kann.

\smallskip

Anschließend wurde ein neuronales Netz als Lösungsansatz vorgestellt, welches in der Lage ist, die Ausrichtung eines Bauteils für Flächen in dem Bauteil zu bestimmen. Dafür wurde ebenfalls ein Verfahren zur Extraktion von Flächen aus Bauteilen beschrieben.

\smallskip

Schließlich wurden Test auf dem beschrieben neuronalen Netz durchgeführt und ihre Ergebnisse wurden ausgewertet. Daraus ergab sich, dass das Netz in der Lage ist, Bauteile mit einer akzeptablen, aber bei weitem nicht perfekten, Genauigkeit auszurichten. 
Eine Abweichung von unter 10 Grad ist im Durchschnitt erreicht worden.
Eine solche Genauigkeit ist nicht optimal, aber dennoch ausreichend, um die Vorteile der Ausrichtung in der additiven Fertigung zu nutzen. Auch auf komplexeren Bauteilen konnte das Netz verlässlich akzeptable Ergebnisse erzielen, 
was auf eine gute Generalisierbarkeit des Netzes hinweist. Eine durchschnittliche Abweichung von 10 Grad ist für den Praxisfall häufig zu viel. 10 Grad reichen aus, um einer Fläche einen Überhang zu geben, was dem Sinn des 5d-Drucks widersprechen würde.
Das Ergebnis, wenn die Minimierung der Bewegung ignoriert wird ist in der Hinsicht vielversprechender mit 5 Grad. Es würde allerdings auch eine signifikante Erhöhung der Druckzeit mit sich bringen, da sich die Bewegung für jede Ausrichtung mehr als verdoppelt hat.
Die Ausrichtung ist in beiden Fällen nicht perfekt, für eine praktische Anwendung würde man bessere Ergebnisse wollen, aber Neurale Netz sind dennoch in der Lage eine Ausrichtung mit akzeptabler Abweichung zu finden.

\smallskip

Die Minimierung der Bewegung des Bauteils während des Ausrichtens hatte eine kleine, aber sichtbaren Verkleinerung der benötigten Drehung ergeben. Der durchschnittliche Winkel, um den ein Teil gedreht werden muss, lag bei unter 40 Grad. 
Ohne die Minimierung würde der durchschnittliche Drehwinkel bei über 45 Grad liegen, also zufällig zwischen 0 und 90 Grad.
Auch wenn die Verbesserung durch die Minimierung der Bewegung nicht sehr groß ist, so ist sie dennoch relevant, da jede eingesparte Drehung die Druckzeit verringert. Es ist kein anderes Verhalten bei komplexen Bauteilen zu sehen gewesen.
Insgesamt, ist die Verbesserung über dem Zufall nicht sehr groß, aber eine extreme Verbesserung über dem Fall, in dem die Bewegung des Bauteils ignoriert wurde.

\smallskip

Das Finden von Flächen in Bauteilen hat in den meisten Fällen sehr gut funktioniert. Allerdings besonders bei komplexen Bauteilen, die über viele gebogene Flächen verfügen, hat die Suche häufig Flächen mit Löchern oder extrem kleine Flächen ergeben. 
Die Löcher in den Flächen sind in sich selber kein Problem, die Suche hat ein Loch in der Fläche gelassen, weil dort die Punkte einen zu unterschiedlichen Normalvektor hatten. So sollte die Suche auf funktionieren, 
aber ein Loch in einer Fläche könnte das Nutzen von Stützstrukturen wieder nötig machen, was die Vorteile der Ausrichtung wieder verringert.
Durch den Erfolg bei einfachen Bauteilen, lässt sich schlussfolgern, dass die Flächensuche besonders gut darin ist Flächen die nicht gebogen sind zu finden und die sich durch einen Knick in der Geometrie von anderen Flächen abgrenzen lassen. 

\smallskip

Es lässt sich sagen, dass das beschriebene neurale Netz in der Lage ist, die Ausrichtung von Bauteilen zu erkennen, ob die Abweichung gering genug ist um Praxisrelevanz zu sein ist fragwürdig.  Eine geringere Abweichung von dem Optimum wäre wünschenswert, 
aber die erzielten Ergebnisse sind dennoch ausreichend, 
um die Vorteile der Ausrichtung zu nutzen. Die Flächensuche funktioniert in den meisten Fällen gut, aber es gibt noch Raum für Verbesserungen, besonders bei komplexen Bauteilen mit gebogenen Flächen. Die Minimierung der Bewegung des Bauteils ist funktional, 
aber die Verbesserung ist nur gering. Insgesamt ist das beschriebene Verfahren ein akzeptabler Ansatz, um Bauteile in der additiven Fertigung auszurichten, wenn auch noch einige Verbesserungen möglich wären. 
Es ist zusammengefasst für den praktischen Einsatz unter Umständen praktikabel. Bei einer Abweichung von bis zu 40 Grad ist das Drucken immer noch möglich, wenn auch nicht sehr schnell \cite{jiang2018investigation}.

\section{Ausblick}
\label{Ausblick}

Diese Arbeit hat gezeigt, dass das Ausrichten von Bauteilen mit neuronalen Netzen möglich ist. Es gibt allerdings noch viele Bereiche in denen Verbesserungen vorgenommen werden können.

\smallskip 

Die Genauigkeit des Netzes könnte durch eine Verbesserung der Flächensuche gesteigert werden. Die Flächensuche selber basiert auf einem fehlerbehafteten Ansatz. Für unterschiedliche Arten von Bauteile müssen die Parameter der Flächensuche angepasst werden, 
was dem Sinn von automatischer Anpassung widerspricht. Eine  Flächensuche, die weniger Fehler macht, würde dem Neuralen Netz bessere Trainingsdaten liefern, was die Genauigkeit des Netzes steigern könnte. Ein weiteres Problem der Flächensuche ist, dass sie Flächen mit Löchern findet.
Das ist ein Problem, was mit dem Clustering von Normalvektoren zusammenhängt und dementsprechend vermutlich nicht behoben werden kann, ohne den Ansatz der Flächensuche zu verändern. Eine Flächensuche, die in der Lage ist, Flächen mit Löchern zu vermeiden, 
würde das Problem der Stützstrukturen lösen.

\smallskip

Die Architektur des Netzes könnte ebenfalls verändert werden, um die Genauigkeit zu steigern. Es könnten andere Architekturen ausprobiert werden, die eventuell besser für das Problem geeignet sind. Auch eine Veränderung der Hyperparameter des Netzes könnte die Genauigkeit verbessern.

\smallskip

Weitere Arbeiten könnten dementsprechend sich auf bessere Flächensuche und Architektur des Netzes konzentrieren.



% Anhang ---------------------------------------------------------------
%
\cleardoublepage
\appendix

\include{kapitel/anhang}

% Abbildungsverzeichnis -------------------------------------------------
%
\listoffigures
\ifthenelse{\boolean{spracheDT}}{ % deutsch
\addcontentsline{toc}{chapter}{Abbildungsverzeichnis}
}{ % englisch
\addcontentsline{toc}{chapter}{List of Figures}
}

\cleardoublepage

% Algorithmenverzeichnis ------------------------------------------------
%
\listofalgorithms
\ifthenelse{\boolean{spracheDT}}{ % deutsch
\addcontentsline{toc}{chapter}{Algorithmenverzeichnis}
}{ % englisch
\addcontentsline{toc}{chapter}{List of Algorithms}
}
\cleardoublepage

% Quellcodeverzeichnis --------------------------------------------------
%
\ifthenelse{\boolean{spracheDT}}{ % deutsch
\renewcommand{\lstlistlistingname}{Quellcodeverzeichnis}
\lstlistoflistings
\addcontentsline{toc}{chapter}{Quellcodeverzeichnis}
}{ % englisch
\renewcommand{\lstlistlistingname}{List of Listings}
\lstlistoflistings
\addcontentsline{toc}{chapter}{List of Listings}
}
\cleardoublepage

% Literaturverzeichnis -------------------------------------------------
%
\printbibliography[heading=bibintoc]

% ----------------------------------------------------------------------
% Erklärung zur Verwendung von Hilfsmitteln und KI-gestützte Schreibwerkzeugen
%========================================================================================
% TU Dortmund, Informatik Lehrstuhl VII
%========================================================================================

\ifthenelse{\boolean{spracheDT}}{ % deutsch
\chapter*{Erklärung zur Verwendung von Hilfsmitteln und KI-gestützten Schreibwerkzeugen}

Ich versichere, dass ich beim Einsatz von IT-/KI-gestützten Schreibwerkzeugen diese Werkzeuge in der nachfolgenden Übersicht der verwendeten Hilfsmittel mit 
ihrem Produktnamen und meiner Bezugsquelle vollständig aufgeführt und/oder die betreffenden Textstellen in der Arbeit als mit IT/KI generierter Unterstützung 
verfasst gekennzeichnet habe.\\

Mir ist bewusst, dass Täuschungen bzw. Täuschungsversuche nach der für mich geltenden Prüfungsordnung geahndet werden.\\

Folgende Hilfsmittel und KI-gestützten Schreibwerkzeuge habe ich für die Bearbeitung genutzt:
\begin{itemize}
  \item LanguangeTool, LanguangeTool, 1.0.1, Bezugsquelle, \url{https://languagetool.org/de}
  \item Hilfsmittel, Produktnamen, Version, Bezugsquelle, \url{Webseite}
  \item Hilfsmittel, Produktnamen, Version, Bezugsquelle, \url{Webseite}
\end{itemize}
}{ % englisch
\chapter*{Declaration on the use of tools and AI-assisted writing tools}

Ich versichere, dass ich beim Einsatz von IT-/KI-gestützten Schreibwerkzeugen diese Werkzeuge in der nachfolgenden Übersicht der verwendeten Hilfsmittel mit ihrem Produktnamen und meiner Bezugsquelle vollständig aufgeführt und/oder die betreffenden Textstellen in der Arbeit als mit IT/KI generierter Unterstützung verfasst gekennzeichnet habe.\\

Mir ist bewusst, dass Täuschungen bzw. Täuschungsversuche nach der für mich geltenden Prüfungsordnung geahndet werden.\\

Folgende Hilfsmittel und KI-gestützten Schreibwerkzeuge habe ich für die Bearbeitung genutzt:
\begin{itemize}
  \item LanguangeTool, LanguangeTool, 1.0.1, Bezugsquelle, \url{https://languagetool.org/de}
  \item Hilfsmittel, Produktnamen, Version, Bezugsquelle, \url{Webseite}
  \item Hilfsmittel, Produktnamen, Version, Bezugsquelle, \url{Webseite}
\end{itemize}
} 

% ----------------------------------------------------------------------
% Eidesstattliche Versicherung
% Laden Sie die aktuelle Erklärung vom Dezernat 4.3 herunter, und unterzeichnen Sie diese.
% https://www.tu-dortmund.de/eidesstattliche-versicherung
% Sie können die Datei dann hier per \includepdf{Eidesstattliche_Versicherung.pdf} einbinden

\end{document}
